\begin{section}{Resultados}
	El algoritmo de $Newton$ utilizado para buscar los ceros de un polinomio hace uso de dos parámetros, los cuales tuvimos que ajustar de manera de conseguir los mejores resultados posibles, mejores en el sentido de relación calidad de la solución y eficiencia en terminos de tiempo del algoritmo.
	
	Uno de los parámetros es para determinar la máxima cantidad de iteraciones que le permitimos al algoritmo buscar. Para ajustar este parámetro BALABALABNAKABALKABKANSJBFDFKBFKDBC.

	El siguiente gráfico se realizó para ajustar un parámetro ($tolerancia$) del programa que sirve buscar los $ceros$ de un polinomio. Este parámetro se utiliza para decidir si un valor es cero, es decir, cuando es menor a la $tolerancia$ lo consideramos cero.
	
	Con esta prueba esperamos ver que cuanto menor es la $tolerancia$ mejor es la aproximación al cero teórico. Recordar que el algoritmo que busca los ceros de un polinomio tiene dos criterios de parada (la tolerancia y la cantidad de iteraciones permitidas), creemos que la aproximación va a mejorar al disminuir la tolerancia ya que si termina por el primer motivo el valor conseguido va a ser más refinado (más cercano a 0) y si lo hace por el segundo va a mejorar la aproximación durante más iteraciones.
	
\end{section}

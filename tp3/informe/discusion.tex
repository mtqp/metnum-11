\begin{section}{Discusión}
	En esta sección, buscaremos conclusiones a la información suministrada por los gráficos de la sección anterior. Dividiremos la sección en dos subsecciones analizando en cada una de ella, cada una de las mismas en los resultados.\\
	
\begin{subsection}{Ajuste de parámetros}
	Ambos gráficos están relacionados entre sí, por lo tanto, no analizaremos cada uno por separado. 
	
	Podemos notar en primera instancia, que utilizar como cantidad de iteraciones permitidas mayores a 1000 en el algoritmo, no tiene sentido, ya que las soluciones no mejoran, es decir el error cometido al buscar la raíz no disminuye. Si bien no estamos contemplando todos los casos posibles, ya que existen infinitos polinomios, por ende infinitas soluciones, podemos reforzar esta conclusión viendo que en el segundo gráfico (Figura \ref{fig:tol}) sin importar el costo (donde costo es una medida proporcional a la cantidad de iteraciones) la calidad de la solución no mejora en la casi totalidad de los casos de $1.0e-17$, es decir no se aproxima más al cero ideal.
	Estos nos lleva a inducir que un parámetro para la cantidad de iteraciones máxima de valor $1000$ es suficiente para que el algoritmo aproxime lo más posible a la raíz.
	
	Por otro lado, el primer gráfico nos muestra que las soluciones no mejoran al error de $1.0e-17$. Al ser pocos polinomios los analizados, necesitamos más información para poder inferir una cota para la $tolerancia$. Recolectando información sobre el segundo gráfico, vemos que salvo escasos casos atípicos, el resto de las muestras, no mejoran la misma cota impuesta para el gráfico anterior.
	
	Nuevamente no podemos afirmar que este comportamiento va a ser igual para cualquier instancia presentada. A pesar que el generador de polinomios al azar intenta ser lo mas uniforme posible, puede quizás existir un sesgo que no estamos tomando en cuenta y por lo tanto las soluciones no se pueden mejorar. Otra causa puede ser que el error intrínseco de las operaciones con punto flotante no nos dejen mejorar la solución.
	
	Es arbitraria la decisión tomada, pero considerando el uso que le damos y viendo que en nuestras muestras el error no se hace más chicos que la cota previamente mencionada, tomamos como $tolerancia$ al valor de $1.0e-17$.
	
\end{subsection}
\begin{subsection}{Análisis de curvas}
	Los gráficos referenciados por (Figura:\ref{fig:5p} y Figura:\ref{fig:5p_r}) vemos que la elección de la parametrización define la forma en la que se aproxima la curva. Si bien las tres parametrizaciones cumple todas las condiciones exigidas por los $splines$, la que utiliza la parametrización por longitud de cuerda es la que mejor aproxima a la curva original mientras que la parametrización uniforme consigue la peor aproximación.
	
	Observamos que la parametrización uniforme (basándonos en estas instancias) tiende a unir los puntos con lineas más rectas cuanto más separados estos se encuentran. Es decir si existe cambios abruptos en la dirección donde la curva debe moverse y además en el lugar donde se produce este cambio, existe varios puntos cercanos entre sí, se generan oscilaciones importantes.  
	
	%Si tomamos el par de puntos de control que se encuentran entre $-5$ y $-4$ vemos que todas ellas se asemejan entre sí. La uniforme sin embargo se separa de la curva original debido a que el punto de control siguiente se encuentra alejado de ella y prácticamente en el mismo $y$.
	
	Por lo que podemos ver, la parametrización uniforme no modela bien las curvas en los casos donde entre un par de puntos de control la curva original no los une mediante algo que se asemeja a una recta.
	
	Inferimos además que, para la parametrización por longitud de cuerda, no es necesario tener puntos de control muy cercanos entre sí para obtener una buena aproximación, (ver segundo y tercer punto de control).
	
	La parametrización centrípeta si bien es mejor que la uniforme (en estos casos), evitando por ejemplo la oscilación entre los tres puntos de control cercanos entre sí, dista considerablemente de la curva original y de la aproximación a la curva bajo la parametrización por longitud de cuerda.\\
	
	En el otro gráfico donde los puntos de control fueron seleccionados de manera uniforme (Figura:\ref{fig:5p_u}) observamos que a pesar de la proximidad de las $cuatro$ curvas, la parametrización por longitud de cuerda sigue siendo la que mejor aproxima y uniforme la que peor lo hace.
	
	Concluimos finalmente que la parametrización por longitud de cuerda tiene trayectorias suaves en contraste con las demás (centrípeta y uniforme), siendo centripeta a su vez más suave que uniforme. 
	
	Además, la parametrización uniforme es totalmente dependiente de la distribución de los puntos de control mientas que el resto de las parametrizaciones (centrípeta y longitud de cuerda) lo son en menor medida, siendo longitud de cuerda la menos dependiente de este suceso.\\
	
	En los gráficos (Figura:\ref{fig:uniform}, \ref{fig:centripetal}y \ref{fig:chordLength}) podemos ver que a medida que la cantidad de puntos de control aumenta, la aproximación se acerca cada vez más a la curva real. De todas formas, la velocidad de acercamiento varía dependiendo de la parametrización, pudiendo notar (basándonos en la curva analizada), que la parametrización que mayor cantidad de puntos de control necesita para aproximar la curva es la uniforme, seguido de la centrípeta y por último la parametrización por longitud de cuerda.
	
	En el caso de la parametrización por longitud de cuerda, observamos que dada la presición del gráfico, este no aporta información sobre la existencia de diferencias con respecto a la curva real. Por lo tanto, y suponiendo que quisieramos visualizar y/o analizar datos con esa precision, no sería necesario aumentar la cantidad de puntos de control.
	
	\underline{\texttt{Nota:}} Es importante destacar que existen infinitos polinomios en $\mathbb{R}^n$ (en particular en $\mathbb{R}^n$), por lo tanto no podemos considerar como generales las conclusiones tomadas en este análisis, ya que pueden existir casos donde la mejor aproximación a la curva no es utilizando la parametrización por longitud de cuerda. Tanto los gráficos como el análisis nos da una idea general de como estas se comportan.
\end{subsection}


	
\end{section}

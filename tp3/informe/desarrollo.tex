\begin{section}{Desarrollo}
	\begin{subsection}{Explicación}
		Para construir la curva parámetrica se puede utilizar cualquiera de las $tres$ parametrizaciones (uniforme, chord-length y centripeta) en el intervalo
		$[0,1]$.
		La parametrización $t$ se elige a partir de los puntos de control $(x,y)$ recibidos en la entrada del programa.
		Una vez elegida la parametrización se generan dos splines, el primero $S_x$ a partir del par $(t,x)$ y el otro $S_y$ a partir de $(t,y)$.
		De esta manera, queda definida una curva $C \in \mathbb{R}^2$ donde $C(t) = (S_x(t),S_y(t),)$.
		
		Para mover un punto $(x,y)$ cercano a la curva a una nueva posición $(x*,y*)$, calculamos primero el punto de la curva más próximo a $(x,y)$ y luego
		construimos una nueva curva de manera tal que ambas splines ($S_x$ y $S_y$) ahora pasen también por la nueva posición $(x*,y*)$.
		
		Para calcular el punto de la curva más próximo a $(x,y)$ minimizamos (derivamos y buscamos donde se anula distinguiendo entre máximos y mínimos) la 
		función distancia\footnote{Distancia de $(S_x(t),S_y(t))$ a $(x,y)$: $\sqrt{(S_x(t)-x)^2+(S_y(t)-y)^2}$} de la curva a $(x,y)$ en cada intervalo
		$[t_i,t_{i+1}] \in [0,1]$ (polinomio que la conforma) y luego seleccionamos el mínimo en $[0,1]$.
	\end{subsection}
	\begin{subsection}{Implementación}
		La implementación esta divida en módulos que realizan tareas especificas, a continuación detallaremos cada uno de ellos.
		
		\begin{itemize}
			\item \underline{Módulo Parametrización:}\\
				Este módulo implementa las $tres$ parametrizaciones (uniforme,\\
				chord-length, centripeta) en el $[0,1]$ dado un conjunto de puntos de control.
			\item \underline{Módulo Polinomio:}\\
				Escribimos el módulo \texttt{Polinomio} que implementa un polinomio de\\
				grado $n$ con las siguientes operaciones:\\
				
				\begin{tabular}{rl}
					\texttt{Evaluar} & Evalua el polinomio en un valor recibido por parámetro.\\
					\texttt{Derivar} & Realiza la derivada primera del polinomio.\\
					\texttt{Ceros}   & Busca una raíz del polinomio usando bisección y el método\\
									 & de Newton.\\
				\end{tabular}\\ 
				EXPLICAR CON MAYOR DETALLES LAS OPERACIONES (COMO NEWTON)!!!!!!
		
			\item \underline{Módulo Spline:}\\
				Escribimos el módulo \texttt{Spline} que implementa un spline cúbico natural con las siguientes operaciones:\\
				
				\begin{tabular}{rl}
					\texttt{Evaluar} & Evalua la spline (el polinomio correspondiente) en un\\
									 & valor recibido como parámetro\\
									 & como parámetro.\\
					\texttt{Polinomio} & Devuelve el polinomio requerido.\\
				\end{tabular}\\

			\item \underline{Módulo Curva:}\\
				El módulo \texttt{Curva} implementa una curva parámetrica con las siguientes operaciones:\\
				
				\begin{tabular}{rl}
					\texttt{Punto cercano} & Dadas las coordenadas de un punto cercano a la curva, calcula\\
										   & el punto de la curva más próximo.\\
					\texttt{Mover punto} & Dadas las coordenadas de un punto cercano a la curva, calcula el\\
										 & punto de la curva más próximo y construye una nueva spline\\
										 & resultante de modificar la spline original de manera que ahora\\
										 & pase por la nueva posición del punto seleccionado.\\
					\texttt{Muestreo} & Devuelve un muestreo de la curva.\\
				\end{tabular}\\
				
				Esta clase cuenta con un método $private$ que busca el $t \in [0,1]$ tal que al evaluar la curva en $t$ se obtiene el punto más cercano a la 
				misma respecto de un punto dado $(x,y)$.
				
				Se busca el $t$ que minimiza la función distancia$^1$
				del punto $(x,y)$ a cada polinomio de la curva en el intervalo correspondiente ($[t_i,t_{i+1}]$) y luego se calcula la distancia de evaluar
				la curva en cada $t$ obtenido (que minimiza la distancia a cada polinomio) a $(x,y)$ y se selecciona el mínimo.
				
				Llamamos $d(t)$ a la distancia de $(S_x(t),S_y(t))$ a $(x,y)$. Para minimizar $d(t)$ obtenemos la derivada y buscamos los $t$ donde se anula. 
				Estos $t$ que son los puntos criticos de la función distancia corresponden a máximos o mínimos. 
				
				
		\end{itemize}
	\end{subsection}
\end{section}

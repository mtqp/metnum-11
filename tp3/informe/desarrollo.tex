\begin{section}{Desarrollo}
	\begin{subsection}{Explicación}
		Para construir la curva parámetrica se puede utilizar cualquiera de las $tres$ parametrizaciones (uniforme, chord-length y centripeta) en el intervalo $[0,1]$.
		La parametrización $t$ se elige a partir de los puntos de control $(x,y)$ recibidos en la entrada del programa.
		Una vez elegida la parametrización se generan dos splines, el primero $S_x$ a partir del par $(t,x)$ y el otro $S_y$ a partir de $(t,y)$.
		De esta manera, queda definida una curva $C \in \mathbb{R}^2$ donde $C(t) = (S_x(t),S_y(t))$.
		
		Para mover un punto $(x,y)$ cercano a la curva a una nueva posición $(x*,y*)$, calculamos primero el punto de la curva más próximo a $(x,y)$ y luego construimos una nueva curva (manteniendo la parametrización original) de manera tal que ambas splines ($S_x$ y $S_y$) ahora pasen también por la nueva posición $(x*,y*)$.
		
		Para calcular el punto de la curva más próximo a $(x,y)$ minimizamos (derivamos y buscamos donde se anula distinguiendo entre máximos y mínimos) la	función distancia$^1$ de la curva al punto $(x,y)$ en cada intervalo $[t_i,t_{i+1}] \in [0,1]$ (polinomio que la conforma) y luego seleccionamos el mínimo en $[0,1]$.\\
		
		A continuación se detalla la busqueda del punto más cercano a la curva dado $(x,y)$.
		
		Sea $n$ la cantidad de puntos de control y $S_x^i$, $S_y^i$ el $i-esimo$ polinomio de cada spline de la curva.\\
		 
		$t /\; min\; \sqrt{(S_x(t)-x)^2+(S_y(t)-y)^2} \left\{
		\begin{array}{c}
		t_1 /\; min\; \sqrt{(S_x^{(1)}(t)-x)^2+(S_y^{(1)}(t)-y)^2}\\
		t_2 /\; min\; \sqrt{(S_x^{(2)}(t)-x)^2+(S_y^{(2)}(t)-y)^2}\\
		\vdots\\
		t_{n-1} /\; min\; \sqrt{(S_x^{(n-1)}(t)-x)^2+(S_y^{(n-1)}(t)-y)^2}\\
		\end{array}
		\right.$
		\VSP		
	\end{subsection}
	\begin{subsection}{Implementación}
		La implementación esta divida en módulos que realizan tareas especificas, a continuación detallaremos cada uno de ellos.
		
		\begin{itemize}
			\item \underline{Módulo Parametrización:}\\
				Este módulo implementa las $tres$ parametrizaciones (uniforme,\\
				chord-length, centripeta) en el $[0,1]$ dado un conjunto de puntos de control.
		
			\item \underline{Módulo Spline:}\\
				Escribimos el módulo \texttt{Spline} que implementa un spline cúbico natural con las siguientes operaciones:\\
				
				\begin{tabular}{rl}
					\texttt{Evaluar} & Evalua la spline (el polinomio correspondiente) en un\\
									 & valor recibido como parámetro\\
					\texttt{Polinomio} & Devuelve el polinomio requerido.\\
				\end{tabular}\\

			\item \underline{Módulo Curva:}\\
				El módulo \texttt{Curva} implementa una curva parámetrica con las siguientes operaciones:\\
				
				\begin{tabular}{rl}
					\texttt{Punto cercano} & Dadas las coordenadas de un punto cercano a la curva, calcula\\
										   & el punto de la curva más próximo.\\
					\texttt{Mover punto} & Dadas las coordenadas de un punto cercano a la curva, calcula el\\
										 & punto de la curva más próximo y construye una nueva spline\\
										 & resultante de modificar la spline original de manera que ahora\\
										 & pase por la nueva posición del punto seleccionado.\\
					\texttt{Muestreo} & Devuelve un muestreo de la curva.\\
				\end{tabular}\\
				
				Esta clase cuenta con un método $private$ (que se usa tanto para \texttt{'Punto cercano'} como para \texttt{'Mover Punto'}) que busca el $t \in [0,1]$ tal que al evaluar la curva en $t$ se obtiene el punto más cercano a la misma respecto de un punto dado $(x,y)$.
				
				Se busca el $t$ que minimiza la función distancia\footnote{Distancia euclídea de $(S_x(t),S_y(t))$ a $(x,y)$: $\sqrt{(S_x(t)-x)^2+(S_y(t)-y)^2}$} del punto $(x,y)$ a cada polinomio de la curva en el intervalo correspondiente ($[t_i,t_{i+1}]$) y se verifica si se puede actualizar el mínimo global, es decir, se selecciona el $t$ que minimiza la distancia en $[t_1,t_{i+1}]$.
				
				Llamamos $d_i(t)$ a la distancia de $(S_x^{(i)}(t),S_y^{(i)}(t))$ a $(x,y)$, es decir, la distancia a la curva en el intervalo $[t_i,t_{i+1}]$ ($i-esimo$ polinomio).\\
				
				Como la función distancia es BLABLABLA podemos minimizar $d_i^2(t)$.\\
				
				$d_i^2(t) = (S_x^{(i)}-x)^2+(S_y^{(i)}-y)^2$\\
				
				Sea $z=(t-t_i) \Rightarrow$
				
				$d_i^2(t) = (a_{x_i} + b_{x_i}*z + c_{x_i}*z^2 + d_{x_i}*z^3 - x)^2 + (a_{y_i} + b_{y_i}*z + c_{y_i}*z^2 + d_{y_i}*z^3 -y)^2$\\
				
				Para minimizar $d_i^2(t)$ obtenemos la derivada:.
				
				$(d_i^2)'(t) = 2(S_x^{(i)}(t)-x)(S_x^{(i)}(t)-x)'+2(S_y^{(i)}(t)-y)(S_y^{(i)}(t)-y)'$\\
				
				Sea $E_{k_i} = 2(S_k^{(i)}(t)-k)(S_k^{(i)}(t)-k)'$ \\
				
				$E_{k_i} = 2b_{k_i}(a_{k_i}-k)+2(b_{k_i}^2+2c_{k_i}(a_{k_i}-k))*z+6(c_{k_i}b_{k_i}+d_{k_i}(a_{k_i}-k)))*z^2+4(2d_{k_i}b_{k_i}+c_{k_i}^2)*z^3+10d_{k_i}c_{k_i}*z^4+6d_{k_i}^2*z^5$\\
				
				$\Rightarrow (d_i^2)'(t) = E_{x_i} + E_{y_i}$\\
				
				\underline{\texttt{Nota:}} Esta expresión esta 'hardcodeada' en una función $private$ que devuelve el polinomio 'distancia' derivado a partir de los coeficientes de $S_x$, $S_y$.
				
				Por último, buscamos los $t$ donde $(d_i^2)'(t)$ se anula. Estos $t$ que son los puntos críticos de la función corresponden a máximos o a mínimos.
				Nos quedamos con el $t$ tal minimiza $d(t)$.
				
				A continuación exponemos el pseudocódigo de esta función con el objetivo de esclarecer su explicación:\\
				
				\begin{pseudo}
					\func{PuntoMásProx}{$curva$,$(x,y)$}\\
					\tab $min\_global = 0$\\
					\tab \FOR $i\gets 1 \TO cantPtosControl-1$\\
					\tab\tab $pol_i = distDerivada(curva,(x,y),i)$\\
					\tab\tab $ptosCriticos = ceros(pol_i,i,i+1)$\\
					\tab\tab $min\_t = minimo(ptosCriticos)$\\
					\tab\tab $dist\_min\_global = dist(S_x(min\_global),S_y(min\_global),x,y)$\\
					\tab\tab $dist\_min\_t = dist(S_x(min\_t),S_y(min\_t),x,y)$\\
					\tab\tab \IF $dist\_min\_t < dist\_min\_global$ \THEN\\
					\tab\tab\tab $dist\_min\_global = dist\_min_t$\\
					\tab \RET $(S_x(min\_global),S_y(min\_global))$\\
				\end{pseudo}
				
				Para mover un punto $(x,y)$	calculamos el punto de la curva más próximo como se explico previamente. Luego, separamos en los siguientes casos:
				
				\begin{itemize}
					\item \underline{Caso punto más cercano a $(x,y)$ es un punto de control:}\\
					
						Se reemplaza el punto de control (aquel que es el punto más próximo a $(x,y)$) por la posición final de $(x,y)$ y se construye una nueva curva a partir de los nuevos puntos de control manteniendo la parametrización.\\
						
						ACA PONER FIGURA!!!
					
					\item \underline{Caso punto más cercano a $(x,y)$ no es un punto de control:}\\
					
						Se agrega como nuevo punto de control $(x*,y*)$ (posición final de $(x,y)$), además se agrega el $t$ correspondiente a la parametrización (la paramerización para el resto de los puntos de control se mantiene).
						Para esto, se selecciona la posición que le corresponde a estos datos según un orden creciente de $t$. Luego, se construye una nueva curva que ahora también pase por $(x*,y*)$.\\
						
						ACA PONER LAS DOS FIGURAS!!!
				\end{itemize}
				
				La última de las operaciones selecciona $m$ puntos de muestreo, donde $m$ es recibido en la entrada.
				Estos puntos corresponden a un muestreo uniforme del rango del parámetro $([0,1])$ incluyendo los extremos.\\
				
				\item \underline{Módulo Polinomio:}\\
				Escribimos el módulo \texttt{Polinomio} que implementa un polinomio de\\
				grado $n$ con las siguientes operaciones:\\
				
				\begin{tabular}{rl}
					\texttt{Evaluar} & Evalua el polinomio en un valor recibido por parámetro.\\
					\texttt{Derivar} & Realiza la derivada primera del polinomio.\\
					\texttt{Ceros}   & Busca una raíz del polinomio usando bisección y el método\\
									 & de Newton.\\
				\end{tabular}\\ 
				
				Para encontrar los ceros de un polinomio en el intervalo $[a,b]$ implementamos un algortimo heurístico que detallamos a continuación:\\
				
				\begin{pseudo}
					\func{ceros}{$polinomio$,$a$,$b$}\\
					\tab $long\_intervalo = (b-a)/grado(polinomio)$\\
					\tab $b = a+long\_intervalo$\\
					\tab \FOR $i\gets 0 \TO grado(polinomio)-1$\\
					\tab\tab\tab $raices[i] = BuscarRaiz(polinomio,a,b)$\\
					\tab\tab\tab $a = b$\\
					\tab\tab\tab $b = b + long\_intervalo$\\
					\tab \RET $raices$\\
				\end{pseudo}
		
				Dado que el polinomio tiene a lo sumo tantas raíces como su grado, dividimos $[a,b]$ en esa cantidad de intervalos, apostando a que
				las mismas se encuentran uniformemente distribuídas, si esto sucede encontramos una raíz en cada intervalo. Para buscar cada una de 
				ellas ejecutamos el método de bisección mientras sea posible (cambie de signo en el intervalo) o hasta acercarnos lo suficiente
				(parámetro definido por nosotros, Ver Pruebas!!!!). En ambos casos el algoritmo aplica en última instancia el método de $Newton$.\\
				En los intervalos donde no hay ninguna raíz el procedimiento es el mismo, el valor conseguido es producto de que el método de 
				$Newton$ utilizó todas las iteraciones permitidas, no alcanzando este valor la tolerancia (parámetro definido por nosotros, Ver Pruebas!!!!) que le otorga la condición de cero del
				polinomio.
				%(recordemos que como las operaciones son en punto flotante y existe error de representación, consideramos que dos valores son el mismo si cumplen que la diferencia es menor a la tolerancia (elegida por nosotros, Ver Pruebas!!!!), es decir, consideramos que un valor es cero si es menor a la tolerancia).\\
				
				\underline{\texttt{Observación:}} Si tenemos intervalos sin raíces implica que no vamos a encontrarlas todas, ya que obtenemos una raíz por intervalo (vamos a encontrar tantas raíces como intervalos con raíces tengamos).\\
				
				Como lo que buscamos es el mínimo global del polinomio en el intervalo $[a,b]$ y los polinomios a los que les buscamos los puntos críticos son de grado 5 $((d_i^2)'(t))$ obtenemos 5 raíces como se explicó previamente. Luego elegimos aquella que al evaluar la curva nos da un valor menor a todas las demás, es por esto que podemos tolerar tener valores que no son considerados raíces ya que serán descartadas o no en el caso de no haber podido hallar la raíz que corresponde al mínimo global y el resto correspondan a máximos (ese es uno de los posibles casos).
		\end{itemize}
	\end{subsection}
	\begin{subsection}{Correctitud de $splines$}
		Con el fin de probar que efectivamente los cálculos sean correctos, se creó la clase $SplineTester$ la cual exporta un constructor que recibe el $spline$ a analizar y un método $isSuccessful$ que devuelve un valor booleano correspondiente a si cumplió con todas las condiciones para ser un spline o no.
		
		Utilizamos $splines$ naturales.\\
		
		Las siguientes son las condiciones que valida el objeto $SplineTester$:
		
		\begin{itemize}
			\item La derivada segunda en los puntos de control bordes es cero (natural).
			\item Los resultado de evaluar el polinomio a izquierda y derecha de un punto de control en si mismo son iguales (en los casos bordes, es decir primer y último punto, se toma el primer y último $spline$ respectivamente).
			\item La primer derivada de cada par de polinomios contiguos evaluadas en el punto de control son iguales (valen las mismas consideraciones para el primer y último punto de control).
			\item La derivada segunda de cada par de polinomios evaluadas en el punto de control que las une son iguales (iguales consideraciones que en el item anterior).
		\end{itemize}
		
	
	\end{subsection}
\end{section}

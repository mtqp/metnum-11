\begin{section}{Introducción teórica}	
	Multitud de problemas de la vida real se rigen por proporciones constantes entre las magnitudes implicadas: procesos físicos, reacciones químicas, costes de materias primas y sus relaciones para formar otros producto, etc.
	Todas estas situaciones admiten de forma natural una descripción matemática a través de sistemas de ecuaciones lineales.
	Además estos sistemas son útiles como una buena aproximación a sistemas más complejos o sistemas de ecuaciones no-lineales.
	
	La resolución de muchos problemas conlleva el manejo de grandes sistemas de ecuaciones lineales, por lo que es necesario plantearse métodos eficientes para su análisis.
	
	Los sistemas de ecuaciones lineales son un conjunto de $m$ ecuaciones relacionadas en $n$ variables. Se los puede expresar en forma matricial como $A \dot x = b$ donde $A$ es una matriz de $m \times n$, $x$ es un vector columna de tamaño $n$ y $b$ un vector de tamaño $m$.
	
	Existen diversos métodos para la resolución de estos sistemas de ecuaciones. Haremos una breve introducción a los distintos métodos.
	
	Tenemos por un lado los métodos iterativos como es el caso del método de Jacobi, Gauss-Seidel o del Gradiente Conjugado que como su nombre lo indica se los aplica en forma iterativa para lograr una aproximación a la solución real del sistema en cuestión
	
	Por otra parte, existen los denominados métodos directos entre los cuales se encuentra el método de eliminación de Gauss el cual es una forma directa para llegar en un número finito de pasos a un sistema equivalente pero más simple, la factorización LU que se vale de este último, la descomposición de Cholesky, y la descomposición QR.	\footnote{Burden y J.D.Faires, Análisis numérico, International Thomson Editors, 1998}
	
	\begin{subsection}{Conceptos que utilizaremos}
		\begin{subsubsection}{Números de condición y matrices mal condicionadas:}
			Para saber el condicionamiento de una matriz se utiliza el denominado número de condición, que esta definido como: $K(A) = \left||A \right|| * \left||A^{-1} \right||$ donde $\left|| . \right||$ es una norma matricial.
			Si $K(A)$ es un número chico, entonces pequeñas variaciones en los coeficientes conllevan a pequeña variación en el resultado. En caso contrario, pequeñas variaciones en la matriz produce diferencias muy grandes en la solución del sistema.
			
			Como $K(A) \geq 1$, entonces mientras más cerca de 1 se encuentre, mejor condicionada va a estar.\\
			
			Como en algunos casos puede ser medio tedioso calcular la inversa de una matriz, otra manera de ver el condicionamiento es utilizando la siguiente propiedad:\\

			Sean $x \in \mathbb{R}^n$, $A \in \mathbb{R}^{nxn}$ inversible y $B \in \mathbb{R}^{nxn}$ no inversible entonces:\\
			
			\begin{center} $\frac{1}{K(A)} \leq \frac{\left|| A-B \right||}{\left|| A \right||}$ \end{center}
	
			donde podemos utilizar cualquier matriz no inversible y de ese modo, ver con alguna norma matricial sencilla (como la norma infinito o la norma 1) el condicionamiento de una determinada matriz.
			
			Otro factor que podemos utilizar también para ver el condicionamiento de una matriz, es calcular el determinante y ver cuán cerca de 0 se encuentra. De ese modo, podemos decir que, mientras más cerca de ser singular esté, peor condicionada va a estar.
		\end{subsubsection}
		\begin{subsubsection}{Matriz de Hilbert}		
			La matriz de Hilbert tiene la característica de ser muy mal condicionada.
			
			Los elementos de dicha matriz se pueden definir como: \\
			
			$H_{ij} = \frac{1}{i+j-1}$\\
			
			Por ejemplo, esta es la matriz de Hilbert de 5x5:
			\[H = \left( \begin{array}{lcccr}
									1      & \frac{1}{2}    & \frac{1}{3} & \frac{1}{4} &  \frac{1}{5}\\
						  \frac{1}{2}      & \frac{1}{3}    & \frac{1}{4} & \frac{1}{5} &  \frac{1}{6}\\
						  \frac{1}{3}      & \frac{1}{4}    & \frac{1}{5} & \frac{1}{6} &  \frac{1}{7}\\
						  \frac{1}{4}      & \frac{1}{5}    & \frac{1}{6} & \frac{1}{7} &  \frac{1}{8}\\
						  \frac{1}{5}      & \frac{1}{6}    & \frac{1}{7} & \frac{1}{8} &  \frac{1}{9}\\
					   \end{array}
				\right)
			\]

			Las matrices de Hilbert son simétricas y definidas positivas, y una cosa que las hace interesantes para estudiar es que su número de condición crece con orden.
		\end{subsubsection}
	\end{subsection}
\end{section}

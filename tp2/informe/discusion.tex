\begin{section}{Discusión}
	En esta sección, buscaremos conclusiones a la información suministrada por los gráficos de la sección anterior.\\
	
	El primer gráfico presenta la calidad de las matrices mal condicionadas generadas fijando un epsilon (Figura:\ref{fig:epsilon}).
	
	Observamos que a medida que fijamos un $epsilon$ menor, obtenemos una matriz con un número de condición mayor. Más aun, el disminuir un orden de magnitud al $epsilon$ implica un aumento en un orden de magnitud en el número de condición de la matriz.
	
	Este gráfico dio sustento a nuestra hipotesis presentada anteriormente (a menor $epsilon$ mayor número de condición).
	
	A partir del gráfico podemos decir también que cuanto mayor sea la dimensión de la  matriz peor condicionada estará (al menos construyendolas de esta manera) ya que los valores obtenidos (número de condición) en función de la dimensión de la matiz, fijando un $epsilon$, forman una curva logarítmica. Como el gráfico se encuentra bajo una escala logarítmica el número de condición de la matriz crece linealmente conforme aumenta su dimensión.
\end{section}

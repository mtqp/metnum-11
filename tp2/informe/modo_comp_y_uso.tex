\begin{section}{Modo de compilación y uso}
		Si se está trabajando sobre un sistema operativo del tipo Unix (Ubuntu en nuestro caso), para compilar se hace uso de la herramienta Makefile.			
		Abrir una terminal dentro la carpeta $code$ entregada y escribir el comando "make", pulsar enter.
		
		En caso de estar trabajando con Windows, se preparó un projecto en la herramienta Codeblocks que se adjunta en la carpeta $code\_win$.
		Abrir el proyecto y pulsar la tecla $F9$, o hacer click en el botón de compilar.
		
		Para ambos sistemas operativos, una vez compilado escriba en la terminal (posicionado sobre la misma ruta en la que lo compiló) el nombre del archivo incluyéndole todos los parámetros necesarios (precondicionados en el enunciado del trabajo práctico).
\end{section}
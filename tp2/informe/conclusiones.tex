\begin{section}{Conclusiones}
	La arítmetica finita utilizada en las computadoras juega un rol esencial en nuestra batalla por dominar el hiperespacio. La propiedad y precondicón de la matriz utilizada para los cálculos (es inversible) nos asegura que existe solución única. Si tuvieramos una aritmética infinita para los cálculos, la batalla no tendría sentido, ya que usando métodos directos, a lo sumo en el segundo ataque hallaríamos al oponente.
	 
	Dado que esto no es así, y aprovechándonos de esta situación, toda la estrategia de la batalla fue centrada en lograr que los cálculos asistidos por computadora del adversario generen suficiente error para no ser descubiertos.
	
	Conocer propiedades sobre las matrices nos ayuda a no caer en falsos positivos, o por lo menos a tener cuidado sobre la información con la que uno está trabajando.
	
	En todas nuestras estrategias vimos en la práctica que no eran lo suficientemente buenas, en particular la de encontrar al enemigo. Hemos visto en esta materia varios métodos de refinamiento de la solución los cuales hubieran sido pertinentes probar. No surgieron diferentes enfoques (a los expuestos en el informe) sobre la forma de obtener la posición del contraincante ya que, el manejo de los datos como así las operaciones y todo cálculo necesario para un correcto funcionamiento del cañon $warp$, lo imposibilitaron (debido a la cantidad de tiempo insumido).
\end{section}

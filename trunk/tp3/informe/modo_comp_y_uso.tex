\begin{section}{Modo de compilación y uso}
	Para compilar se hace uso de la herramienta Makefile.
	
	Abrir una terminal dentro la carpeta $code$ entregada y escribir el comando "make".
	
	Para ejecutar el programa: \texttt{./tp3 input output parametrization\_mode} (donde el archivo input cumple las condiciones del enunciado). 
	
	El tercer parámetro corresponde a los distintos tipos de parametrizaciones, acepta tres valores posibles, $u$ que corresponde a parametrización uniforme, $cl$ que corresponde a $chord$ $length$ y $cp$ que corresponde a centrípeta. En caso de equivocarse con este parámetro e ingresar otro tomará por defecto la parametrización uniforme.
\end{section}
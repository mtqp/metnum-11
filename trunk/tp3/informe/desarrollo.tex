\begin{section}{Desarrollo}
	\begin{subsection}{Explicación}
		Para construir la curva parámetrica se puede utilizar cualquiera de las $tres$ parametrizaciones (uniforme, chord-length y centripeta).
		La parametrización $t$ se elige a partir de los puntos de control $(x,y)$ recibidos en la entrada del programa.
		Una vez elegida la parametrización se generan dos splines, el primero $Sx$ a partir del par $(t,x)$ y el otro $Sy$ a partir de $(t,y)$.
		De esta manera, queda definida una curva $C \in \mathbb{R}^2$ donde $C(t) = (Sx(t),Sy(t),)$.
		
		
	\end{subsection}
	\begin{subsection}{Implementación}
		La implementación esta divida en módulos que realizan tareas especificas, a continuación detallaremos cada uno de ellos.
		
		\begin{itemize}
			\item \underline{Módulo Parametrización:}\\
				Este módulo implementa las $tres$ parametrizaciones (uniforme, chord-length, centripeta) a partir de un conjunto de puntos de control.
			\item \underline{Módulo Polinomio:}\\
				Escribimos el módulo \texttt{Polinomio} que implementa un polinomio de grado $n$ con las siguientes operaciones:\\
				
				\begin{tabular}{rl}
					\texttt{Evaluar} & Evalua el polinomio en un valor recibido por parámetro.\\
					\texttt{Derivar} & Realiza la derivada primera del polinomio.\\
					\texttt{Ceros}   & Busca una raíz del polinomio usando bisección y el método de Newton.\\
				\end{tabular}\\
				EXPLICAR CON MAYOR DETALLES LAS OPERACIONES (COMO NEWTON)!!!!!!
		
			\item \underline{Módulo Spline:}\\
				Escribimos el módulo \texttt{Spline} que implementa un spline cúbico natural con las siguientes operaciones:\\
				
				\begin{tabular}{rl}
					\texttt{Evaluar} & Evalua la spline (el polinomio correspondiente) en un valor recibido como parámetro.\\
					\texttt{Polinomio} & Devuelve el polinomio requerido.\\
				\end{tabular}\\

			\item \underline{Módulo Curva:}\\
				El módulo \texttt{Curva} implementa una curva parámetrica con las siguientes operaciones:\\
				
				\begin{tabular}{rl}
					\texttt{Punto} & Dadas las coordenadas de un punto cercano a la curva, calcula\\
								   & el punto de la curva más próximo.\\
					\texttt{Mover punto} & Dadas las coordenadas de un punto cercano a la curva, calcula el\\
										 & punto de la curva más próximo y construye una nueva spline\\
										 & resultante de modificar la spline original de manera que ahora pase por\\
										 & la nueva posición del punto seleccionado.\\
					\texttt{Muestreo} & Devuelve un muestreo de la curva.\\
				\end{tabular}\\
		\end{itemize}
	\end{subsection}
\end{section}

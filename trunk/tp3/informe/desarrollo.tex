\begin{section}{Desarrollo}
	\begin{subsection}{Explicación}
		
	\end{subsection}
	\begin{subsection}{Implementación}
		La implementación esta divida en módulos que realizan tareas especificas, a continuación detallaremos cada uno de ellos.
		
		\begin{itemize}
			\item \underline{Módulo Parametrización:}\\
				Este módulo implementa las $tres$ parametrizaciones (uniforme, chord-length, centripeta) a partir de un conjunto de puntos de control.
			\item \underline{Módulo Polinomio:}\\
				Escribimos el módulo \texttt{Polinomio} que implementa un polinomio de grado $n$ con las siguientes operaciones:\\
				
				\begin{tabular}{rl}
					\texttt{Evaluar} & Evalua el polinomio en un valor recibido como parámetro.\\
					\texttt{Derivar} & Realiza la derivada primera del polinomio.\\
					\texttt{Ceros}   & Busca una raíz del polinomio usando bisección y el método de Newton.\\
				\end{tabular}\\
				EXPLICAR CON MAYOR DETALLES LAS OPERACIONES (COMO NEWTON)!!!!!!
		
			\item \underline{Módulo Spline:}\\
				Escribimos el módulo \texttt{Spline} que implementa un spline cúbico natural con las siguientes operaciones:\\
				
				\begin{tabular}{rl}
					\texttt{Evaluar} & Evalua la spline (el polinomio correspondiente) en un valor recibido como parámetro.\\
					\texttt{Polinomio} & Devuelve el polinomio requerido.\\
				\end{tabular}\\

			\item \underline{Módulo Curva:}\\
				El módulo \texttt{curva} implementa una curva parámetrica con las siguientes operaciones:\\
				
				\begin{tabular}{rl}
					\texttt{Evaluar} & Evalua la spline (el polinomio correspondiente) en un valor recibido como parámetro.\\
					\texttt{Polinomio} & Devuelve el $n-esimo$ polinomio (siendo $n$ el parámetro).\\
				\end{tabular}\\
		\end{itemize}
	\end{subsection}
\end{section}

\begin{section}{Introducción teórica}	
	 En este trabajo se utilizan splines cúbicos naturales como método de interpolación para generar una curva.\\
	 
	 Un spline es una curva definida en porciones mediante polinomios (en este caso de grado 3). La idea central es que en vez de usar un único polinomio para interpolar todos los datos,
	 se usan segmentos de polinomios entre puntos de control (pares coordenados) y se une cada uno de ellos adecuadamente para ajustar los datos.
	 Los polinomios que definen la curva satisfacen ciertas condiciones específicas de continuidad en la frontera de cada intervalo para asegurar una
	 transición suave.
	 
	 Se utiliza a menudo la interpolación mediante splines porque da lugar a buenos resultados requiriendo solamente el uso de polinomios de bajo grado,
	 evitando así las oscilaciones, indeseables en la mayoría de las aplicaciones, encontradas al interpolar mediante polinomios de grado elevado.\\
	 
	 En este trabajo se utilizan curvas paramétricas en $\mathbb{R}^2$ que dadas las coordenadas de los puntos de control definen la parametrización de la
	 siguiente manera:
	 
	 \begin{description}
		\setlength{\itemsep}{0pt}
		\setlength{\parskip}{0pt}
		\setlength{\parsep}{0pt}
		\item[Uniforme:] la variaci\'on del par\'ametro es igual entre cualquier par de puntos de control consecutivos;
		\item[\emph{Chord-length}:] la variaci\'on del par\'ametro entre dos puntos de control consecutivos es proporcional a la distancia entre los mismos;
		\item[Centr\'ipeta:] la variaci\'on del par\'ametro es proporcional a la ra\'iz cuadrada de la distancia entre los puntos de control.
	\end{description}
	 
\end{section}

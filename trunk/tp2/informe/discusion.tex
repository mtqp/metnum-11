\begin{section}{Discusión}
	En esta sección, buscaremos conclusiones a la información suministrada por los gráficos de la sección anterior.\\
	
	El primer gráfico presenta la calidad de las matrices mal condicionadas en función de su dimensión generadas a 
	partir de un epsilon dado (Figura:\ref{fig:epsilon}).
	
	Observamos que a medida que fijamos un $epsilon$ menor, obtenemos una matriz con un número de condición mayor. 
	Más aun, el disminuir un orden de magnitud al $epsilon$ implica un aumento en un orden de magnitud en el número
	 de condición de la matriz.
	
	Este gráfico dio sustento a nuestra hipotesis presentada anteriormente (a menor $epsilon$ mayor número de 
	condición).
	
	A partir del gráfico podemos decir también que cuanto mayor sea la dimensión de la  matriz peor condicionada 
	estará (al menos construyéndolas de esta manera) ya que los valores obtenidos (número de condición) en función 
	de la dimensión de la matiz, fijando un $epsilon$, forman una curva logarítmica. Como el gráfico se encuentra 
	bajo una escala logarítmica el número de condición de la matriz crece linealmente conforme aumenta su dimensión.
	
	Teniendo en cuenta los resultados aquí expuestos y considerando las restricciones impuestas en el valor de 
	$epsilon$ (ver sección \texttt{Resultados}) decidimos que $1e^{-6}$ es el valor adecuado.
	
	Analizando el segundo gráfico, vemos que la implementación utilizando el algoritmo de factorización LU conseguimos soluciones más aproximadas a la exacta. PORQUEPORQUEPORQUEPORQUEPORQUEPORQUEPORQUEPORQUEPORQUEPORQUEPORQUEPORQUEPORQUEPORQUEPORQUEPORQU
	EPORQUEPORQUEPORQUEPORQUEPORQUEPORQUEPORQUEPORQUEPORQUEPORQUEPORQUEPORQUEPO
	RQUEPORQUEPORQUEPORQUEPORQUEPORQUEPORQUEPORQUEPORQUEPORQUEPORQUEPORQUEPORQUEPOR
	QUEPORQUEPORQUEPORQUEPORQUEPORQUE
\end{section}

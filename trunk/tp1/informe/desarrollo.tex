\begin{section}{Desarrollo}
	En este trabajo práctico aproximamos $\pi$ evaluando hasta un término dado la \texttt{Serie de Gregory}, la \texttt{Fórmula de Machin} y la \texttt{Serie de Ramanujan}.
	Como dichos métodos se implementan mediante aritmética finita de $t$ dígitos pretendemos analizar el comportamiento numérico de cada uno de estos métodos.
	 
	\begin{subsection}{Analisis teórico}
		A continuación analizamos la propagación del error de cada una de las series en el tercer término (en función de los errores de los primeros).\\
		
		Para simplificar dicha tarea calculamos previamente el error de operaciones comunes a las $tres$ series (division, suma, resta y multiplicación).\\
		
		Sea $\funcFull{div(x,y)} {\frac{x}{y}}$.\\
		
		$\e{div}{x,y,:}{\frac{\frac{x}{y}\ev{x}}{f} + \frac{\frac{-xy}{y^2}\ev{y}}{f} + \er{:}{x,y}} = 
		\frac{\frac{x}{y}\ev{x}}{\frac{x}{y}} + \frac{\frac{-x}{y}\ev{y}}{\frac{x}{y}} + \er{:}{x,y} = \ev{x} - \ev{y} + \er{:}{x,y}$\\
		
		Sea $\funcFull{sum(x,y)}{x+y}$\\
	
		$\e{sum}{x,y,+} = \frac{x}{x+y}\ev{x} + \frac{y}{x+y}\ev{y} + \er{+}{x,y}$\\
		
		Sea $\funcFull{resta(x,y)}{x-y}$\\
	
		$\e{resta}{x,y,-} = \frac{x}{x-y}\ev{x} - \frac{y}{x-y}\ev{y} + \er{-}{x,y}$\\
		
		Sea $\funcFull{mult(x,y)}{x*y}$\\
	
		$\e{mult}{x,y,*} = \frac{y*x}{x*y}\ev{x} + \frac{x*y}{x*y}\ev{y} + \er{*}{x,y} = \ev{x} + \ev{y} + \er{*}{x,y}$\\
	
		\underline{Nota:} Realizamos la mayor cantidad de operaciones posibles en el tipo de datos $entero$ ya que de esta manera no poseen error, es decir, son exactas.
		Asumimos que el algoritmo nunca se va a ejecutar con una cantidad de iteraciones tal que desborde.\\
		
		\begin{subsubsection}{Serie de Gregory}
		\begin{equation*}
		\frac{\pi}{4} = \sum_{n=0}^{\infty} \frac{\left(-1\right)^n}{2n+1}
		\end{equation*}

	\VSP
	Definimos primero el error de un término de la serie.\\
		
	Dado $n \in \N$, sea $x=1$ e $y=2n+1$.\\
		
	$\e{div}{x,y,:} = \ev{x} - \ev{y} + \er{:}{x,y}$ como vimos previamente.\\
	
	Este error corresponde al error de un término de la serie de Gregory porque $x$ e $y$ son de tipo entero (exactos), en este caso con $n \leq 2$ lo que implica $y \leq 5$ ($x$ constante) no hay $overflow$.\\
	
	%\underline{Nota:} El error de representación de $x$ e $y$ (al convertirlos a punto flotante para operar) será considerado en el cálculo de error de $\func{f}{x,y}$.\\
	
	Teniendo esta información, llamamos $a$ $(n=0)$ al primer término de la serie, $b$ $(n=1)$ al segundo y $c$ $(n=2)$ al tercero.
	
	Dado que tanto $a$ como $c$ corresponden a $n$ par son términos tal que $(-1)^n = 1$ en contraste con $b$ que tiene signo negativo. Por este motivo,\\
	tenemos que la serie de \texttt{Gregory} hasta el tercer término es: $a - b + c$.\\
	
	Analizamos primero el error cometido de realizar $a+c$.\\
	
	$\e{sum}{a,c,+} = \frac{a}{a+c}\ev{a} + \frac{c}{a+c}\ev{c} + \er{+}{a,c}$\\
	
	Analizamos ahora el error de restar el segundo término ($b$).\\
		
	$\e{restar}{a+c,b,-} = \frac{a+c}{a+c-b}\er{+}{a,c} - \frac{b}{a+c-b}\ev{b} + \er{-}{a+c,b} =$\\
	
	$\frac{a+c}{a+c-b}(\frac{a}{a+c}\ev{a} + \frac{c}{a+c}\ev{c} + \er{+}{a,c}) - \frac{b}{a+c-b}\ev{b} + \er{-}{a+c,b} =$\\
	
	$\frac{a\ev{a} + c\ev{c} - b\ev{b} + (a+c)\er{+}{a,c}}{a-b+c} + \er{-}{a+c,b} = \ev{a+c-b} = \ev{\frac{\pi}{4}_2}$\\
	
	Entonces el error en el tercer término ($\ev{\pi_2}$) es el siguiente:\\
	
	$\e{mult}{4,a+c-b} = \ev{4} + \ev{a+c-b} + \er{*}{4,a+c-b} = \ev{4} + \frac{a\ev{a} + c\ev{c} - b\ev{b} + (a+c)\er{+}{a,c}}{a-b+c} + \er{-}{a+c,b} + \er{*}{4,a+c-b}$\\
	
	Dado que $a=1$, $b=\frac{1}{3}$ y $c=\frac{1}{5}$. Reemplazando por los términos\\
	correspondientes obtenemos:\\
	
	$\ev{4} + \frac{\ev{1} - \ev{1} + \er{:}{1,1} - (\frac{1}{3})(\ev{1} - \ev{3} + \er{:}{1,3}) + (\frac{1}{5})(\ev{1} - \ev{5} + \er{:}{1,5}) + \frac{6}{5} \er{+}{1,\frac{1}{5}}}{\frac{13}{15}} + \er{-}{\frac{6}{5},\frac{1}{3}} + \er{*}{4,\frac{13}{15}}$\\
	
	Como $\ev{1}=0$ ya que al tener un bit implícito no es necesario que la precisión sea mayor a $uno$ $\Rightarrow$\\
	
	$\ev{4} + \frac{\er{:}{1,1} - (\frac{1}{3})(- \ev{3} + \er{:}{1,3}) + (\frac{1}{5})(- \ev{5} + \er{:}{1,5}) + \frac{6}{5} \er{+}{1,\frac{1}{5}}}{\frac{13}{15}} + \er{-}{\frac{6}{5},\frac{1}{3}} + \er{*}{4,\frac{13}{15}}$\\
	
	$\Rightarrow \ev{\pi_2} = \left| \ev{4} + \frac{15\er{:}{1,1} + 5\ev{3} - 5\er{:}{1,3}) - 3\ev{5} + 3\er{:}{1,5}) + 18\er{+}{1,\frac{1}{5}}}{13} + \er{-}{\frac{6}{5},\frac{1}{3}} + \er{*}{4,\frac{13}{15}} \right|$\\
	
	$\leq \left|\ev{4}\right| + \frac{ \left|15\er{:}{1,1}\right| + \left|5\ev{3}\right| + \left|5\er{:}{1,3})\right| + \left|3\ev{5}\right| + \left|3\er{:}{1,5})\right| + \left|18\er{+}{1,\frac{1}{5}}\right|}{13} + \left|\er{-}{\frac{6}{5} ,\frac{1}{3}}\right| + \left|\er{*}{4,\frac{13}{15}}\right|$\\
		
	Si bien no conocemos cuál es el error exacto que posee cada operación, podemos acotarlas por el error máximo de todas ellas en módulo, llamémoslo $u$. Por lo tanto el resultado obtenido es menor igual a:\\

	$u + \frac{15u + 5u + 5u + 3u + 3u + 18u}{13} + u + u = \frac{49}{13}u + 3u = \frac{88}{13}u < 7u$\\

	Sabiendo que la implementación de estos algoritmos será realizada en una máquina con arítmetica binarias, podemos inferir que el error en los cálculos se producirá a partir de cierto bit. El error entonces depende de la conversión de los valores enteros a números con coma, así como también la presición que estemos utilizando. Tomamos de esta forma a $u=2^{1-t}$.
	
	Acotando el error de Gregory, tenemos que $\ev{\pi_2} < 7*2^{1-t}$\\
	
	\underline{Nota:} Este último resultado (el valor de $u$) aplica tanto al análisis previamente realizado, como a los dos subsiguientes (algoritmos de Machin y Ramanujan). Para simplificar los cálculos, y disminuir la probabilidad de arrastrar errores de cuentas, se acotará por $u$ el que luego será reemplazado por $2^{1-t}$.
\end{subsubsection}
		\newpage
		\begin{subsection}{Fórmula de Machin}

	\begin{equation*}
		\frac{\pi}{4} = 4 \; \mathrm{arctan}(1/5) - \mathrm{arctan}(1/239), \text{con } \mathrm{arctan}(x) = \sum_{n=0}^{\infty} \left(-1\right)^n \frac{x^{2n+1}}{2n+1} \quad \text{cuando } \left|x\right|<1
    \end{equation*}
	
	Definimos primero el error de calcular $arctan(k)$ con $|k|<1$. Para ello definimos el error de cada uno de los términos de la sumatoria.\\
	
	Dado $n \in \N$, sea $x=k^{2n+1}$ e $y=2n+1$.\\
	
	$\e{div}{x,y,:} = \ev{x} - \ev{y} + \er{:}{x,y}$ (1).\\
	
	Como $y$ es de tipo entero (exacto) ya que $n \leq 2$ lo que implica $y \leq 5$ no hay $overflow$. Resta calcular el error de $x$.\\
	
	\underline{Analicemos el error de $x=k^y$ ($k=\frac{1}{5}$ o $k=\frac{1}{239} \Rightarrow k>0$):}\\
	
	$\funcFull{$\ev{x}$}{\frac{yk^{y-1}k}{k^y}\ev{k}+\frac{k^y ln(k) y}{k^y}\ev{y} + \er{POT}{k,y}} = y\ev{k} + ln(k) y\ev{y} + \er{POT}{k,y}$ (2)\\ 
	
	$\Rightarrow_{Reemplazando \; (2) \; en \; (1)}$\\
	
	Tenemos entonces que el error de un término de la arcotangente es:\\
	
	$y\ev{k} + ln(k) y\ev{y} + \er{POT}{k,y} - \ev{y} + \er{:}{x,y}$\\
	
	Para acotar el error de un término de la arcotangente empezamos por acotar el error de $k$.
	
	\pa
	
	$\left|\ev{k}\right| = \left\{ 
	\begin{array}{l}
	\left|\ev{\frac{1}{5}}\right| \leq \left|\ev{1}\right| + \left|\ev{5}\right| + \left|\er{:}{1,5}\right| = \left|\ev{5}\right| + \left|\er{:}{1,5}\right|\\
	\left|\ev{\frac{1}{239}}\right| \leq \left|\ev{1}\right| + \left|\ev{239}\right| + \left|\er{:}{1,239}\right| = \left|\ev{239}\right| + \left|\er{:}{1,239}\right|\\
	\end{array}
	\leq 2u\\
	\right.$\\
	
	\pa
	
	Por otra parte tenemos:\\
	
	$\left|\ev{\frac{k^y}{y}}\right| \leq \left|y\ev{k}\right| + \left|ln(k)y\ev{y}\right| + \left|\er{POT}{k,y}\right| + \left|\ev{y}\right| + \left|\er{:}{x,y}\right| =$\\
	
	$(2n+1)(\left|\ev{k}\right| + \left|ln(k)\right|\left|\ev{2n+1}\right|) + \left|\er{POT}{k,(2n+1)}\right| + \left|\ev{(2n+1)}\right| + \left|\er{:}{k^{2n+1},2n+1}\right|$\\
	
	$\Rightarrow \left|\ev{\frac{k^{2n+1}}{2n+1}}\right| \leq (2n+1)(2u + \left|ln(k)\right|u) + 3u = [(2n+1)(2 + \left|ln(k)\right|) + 3]u$\\
	
	donde $u = max\left\{\left|\ev{5}\right|,\;\left|\er{:}{1,5}\right|,\;\left|\ev{239}\right|,\;\left|\er{:}{1,239}\right|,\;\left|\ev{2n+1}\right|,\;\left|\er{POT}{k,(2n+1)}\right|,\;\left|\er{:}{k^{2n+1},2n+1}\right|\right\}$.\\
	
	\pa
	
	Teniendo esta información, queremos realizar el análisis teórico para los primeros tres términos de la sumatoria. Para poder realizarlo, llamamos $a$ $(n=0)$ al primer término,
	$b$ $(n=1)$ al segundo y $c$ $(n=2)$ al tercero.
	
	Dado que tanto $a$ como $c$ corresponden a $n$ par son términos tal que $(-1)^n = 1$ en contraste con $b$ que tiene signo negativo. Por este motivo,\\
	tenemos que la arcotangente hasta el tercer término es: $a - b + c$.\\
	
	Analizamos primero el error cometido de realizar $a+c$.\\
	
	$\e{sum}{a,c,+} = \frac{a}{a+c}\ev{a} + \frac{c}{a+c}\ev{c} + \er{+}{a,c}$\\
	
	Analizamos ahora el error de restar el segundo término ($b$).\\
		
	$\e{restar}{a+c,b,-} = \frac{a+c}{a+c-b}\er{+}{a,c} - \frac{b}{a+c-b}\ev{b} + \er{-}{a+c,b} =$\\
	
	$\frac{a+c}{a+c-b}(\frac{a}{a+c}\ev{a} + \frac{c}{a+c}\ev{c} + \er{+}{a,c}) - \frac{b}{a+c-b}\ev{b} + \er{-}{a+c,b} =$\\
	
	$\frac{a\ev{a} + c\ev{c} - b\ev{b} + (a+c)\er{+}{a,c}}{a-b+c} + \er{-}{a+c,b} = \ev{a+c-b} = \ev{arctan_2}$\\
	
	$\left|\ev{arctan_2}\right| \leq \frac{|a||\ev{a}| + |c||\ev{c}| + |b||\ev{b}| + |a+c|\left|\er{+}{a,c}\right|}{|a-b+c|} + \left|\er{-}{a+c,b}\right|$\\
	
	\underline{Como vimos antes:}
	
	$\left|\ev{a}\right| \leq [(2 + \left|ln(k)\right|) + 3]u$
	
	$\left|\ev{b}\right| \leq [3(2 + \left|ln(k)\right|) + 3]u$
	
	$\left|\ev{c}\right| \leq [5(2 + \left|ln(k)\right|) + 3]u$\\
	
	$\Rightarrow$\\
	$\left|\ev{arctan_2(k)}\right| \leq \frac{|a|[(2 + \left|ln(k)\right|) + 3]u + |c|[5(2 + \left|ln(k)\right|) + 3]u + |b|[3(2 + \left|ln(k)\right|) + 3]u + |a+c|\left|\er{+}{a,c}\right|}{|a-b+c|} + \left|\er{-}{a+c,b}\right|$\\
	
	Para la fórmula de Machin se necesita calcular $arctan(\frac{1}{5})$ y $arctan(\frac{1}{239})$.\\
		
	%$\e{arctan_2}{\frac{1}{5}} = \frac{\frac{1}{5}\ev{a} + (\frac{1}{5})^6\ev{c} - \frac{1}{3}(\frac{1}{5})^3\ev{b} + (\frac{1}{5} + (\frac{1}{5})^5)\er{+}{\frac{1}{5},(\frac{1}{5})^5}}{\frac{1}{5} + (\frac{1}{5})^6 - \frac{1}{3}(\frac{1}{5})^3} + \er{-}{\frac{1}{5} + \frac{1}{5} + (\frac{1}{5})^5,(\frac{1}{5})^3}$\\
	%$\Rightarrow \left| \e{arctan_2}{\frac{1}{5}} \right| \leq \frac{\frac{1}{5}\left| \ev{a} \right| + (\frac{1}{5})^6\left| \ev{c} \right| + \frac{1}{3}(\frac{1}{5})^3\left| \ev{b} \right| + (\frac{1}{5} + (\frac{1}{5})^5)\left| \er{+}{\frac{1}{5},(\frac{1}{5})^5} \right|}{\frac{1}{5} + (\frac{1}{5})^6 - \frac{1}{3}(\frac{1}{5})^3} + \left|\er{-}{\frac{1}{5} + \frac{1}{5} + (\frac{1}{5})^5,(\frac{1}{5})^3} \right|$\\
	%Como vimos antes:\\
	%$\left|\ev{a}\right| \leq [2 + \left|ln(\frac{1}{5})\right| + 3]u \leq [2 + 2 + 3]u \leq 7u$\\
	%$\left|\ev{b}\right| \leq [3(2 + \left|ln(\frac{1}{5})\right|) + 3]u \leq [3(2 + 2) + 3]u \leq 15u$\
	%$\left|\ev{c}\right| \leq [5(2 + \left|ln(\frac{1}{5})\right|) + 3]u \leq [5(2 + 2) + 3]u \leq 23u$\\
	
	$\left|\ev{arctan_2(\frac{1}{5})}\right| \leq \frac{\frac{1}{5}[(2 + 2) + 3]u + \frac{1}{5^6}[5(2 + 2) + 3]u + \frac{1}{3}\frac{1}{5^3}[3(2 + 2) + 3]u + (\frac{1}{5}+\frac{1}{5^6})\left|\er{+}{\frac{1}{5},\frac{1}{5^6}}\right|}{\frac{1}{5}-\frac{1}{3}\frac{1}{5^3}+\frac{1}{5^6}} + \left|\er{-}{\frac{1}{5}+\frac{1}{5^6},\frac{1}{3}\frac{1}{5^3}}\right| =$\\
	
	$\frac{\frac{7}{5}u + \frac{23}{5^6}u +\frac{1}{5}^2u + (\frac{1}{5} + \frac{1}{5^6})\left|\er{+}{\frac{1}{5},\frac{1}{5^6}}\right|}{\frac{1}{5} - \frac{1}{3}\frac{1}{5^3} + \frac{1}{5^6}} + \left|\er{-}{\frac{1}{5} + \frac{1}{5^6},\frac{1}{3}\frac{1}{5^3}} \right|$\\
	
	\pa
	
	Si bien no conocemos cuál es el error exacto que posee cada operación, podemos tomar $u'$ como el error máximo de todas ellas en módulo.\\
	
	$u' =$ máx$\left\{u,\;\left|\er{+}{\frac{1}{5},\frac{1}{5^6}}\right|,\;\left|\er{-}{\frac{1}{5} + \frac{1}{5^6},\frac{1}{3}\frac{1}{5^3}}\right|\right\}$.\\
	
	\pa
	
	$\left| \e{arctan_2}{\frac{1}{5}} \right| \leq \frac{\frac{7}{5}u' + \frac{23}{5^6}u' +\frac{1}{5^2}u' + (\frac{1}{5} + \frac{1}{5^6})u'}{\frac{1}{5} - \frac{1}{3}\frac{1}{5^3} + \frac{1}{5^6}} + u' \leq 10u'$\\
	
	\pa
	
	Por otro lado,\\
	
	$\left|\ev{arctan_2(\frac{1}{239})}\right| \leq \frac{\frac{1}{239}[(2 + 6) + 3]u + \frac{1}{5}\frac{1}{239^5}[5(2 + 6) + 3]u + \frac{1}{3}\frac{1}{239^3}[3(2 + 6) + 3]u + (\frac{1}{239}+\frac{1}{5}\frac{1}{239^5})\left|\er{+}{\frac{1}{239},\frac{1}{5}\frac{1}{239^5}}\right|}{\frac{1}{239}-\frac{1}{3}\frac{1}{239^3}+\frac{1}{5}\frac{1}{239^5}} + \left|\er{-}{\frac{1}{239}+\frac{1}{5}\frac{1}{239^5},\frac{1}{3}\frac{1}{239^3}}\right| =$\\
	
	$\frac{\frac{11}{239}u + \frac{43}{5}\frac{1}{239^5}u + \frac{35}{3}\frac{1}{239^3}u + (\frac{1}{239}+\frac{1}{5}\frac{1}{239^5})\left|\er{+}{\frac{1}{239},\frac{1}{5}\frac{1}{239^5}}\right|}{\frac{1}{239}-\frac{1}{3}\frac{1}{239^3}+\frac{1}{5}\frac{1}{239^5}} + \left|\er{-}{\frac{1}{239}+\frac{1}{5}\frac{1}{239^5},\frac{1}{3}\frac{1}{239^3}}\right|$\\
	
	\pa
	
	Podemos al igual que antes tomar $u''$ como el error máximo en módulo de todas las operaciones involucradas.\\
	
	$\left|\ev{arctan_2(\frac{1}{239})}\right| \leq \frac{\frac{11}{239}u'' + \frac{43}{5}\frac{1}{239^5}u'' + \frac{35}{3}\frac{1}{239^3}u'' + (\frac{1}{239}+\frac{1}{5}\frac{1}{239^5})u''}{\frac{1}{239}-\frac{1}{3}\frac{1}{239^3}+\frac{1}{5}\frac{1}{239^5}} + u'' \leq 13u''$\\
	
	\pa
	
	Finalmente calculamos el error de la fórmula de Machin:
	
	\begin{equation*}
		\frac{\pi}{4} = 4 \; \mathrm{arctan}(1/5) - \mathrm{arctan}(1/239) \Rightarrow \pi = 16 \; \mathrm{arctan}(1/5) - 4 \; \mathrm{arctan}(1/239)
    \end{equation*}
	
	Tomamos $z=16arctan(\frac{1}{5})$ y $w=4arctan(\frac{1}{239})$\\
	
	$\ev{\pi} = \e{restar}{z,w,-} = \frac{z\ev{z} - w\ev{w}}{z-w} + \er{-}{z,w}$\\
	
	$\ev{z} = \e{mult}{16,arctan(\frac{1}{5}),*} = \ev{16} + \ev{arctan(\frac{1}{5})} + \er{*}{16,arctan(\frac{1}{5})}$\\
	
	$\ev{w} = \e{mult}{4,arctan(\frac{1}{239}),*} = \ev{4} + \ev{arctan(\frac{1}{239})} + \er{*}{4,arctan(\frac{1}{239})}$\\
	
	$\Rightarrow \ev{\pi} = \frac{16arctan_2(\frac{1}{5})(\ev{16} + \ev{arctan(\frac{1}{5})} + \er{*}{16,arctan(\frac{1}{5})}) - 4arctan_2(\frac{1}{239})(\ev{4} + \ev{arctan(\frac{1}{239})} + \er{*}{4,arctan(\frac{1}{239})})}{16arctan_2(\frac{1}{5})-4arctan_2(\frac{1}{239})} + \er{-}{16arctan_2(\frac{1}{5}),4arctan_2(\frac{1}{239})}$\\
	
	$\Rightarrow \left|\ev{\pi}\right| \leq \frac{16arctan_2(\frac{1}{5})(|\ev{16}| + 10u' + \left|\er{*}{16,arctan(\frac{1}{5})}\right|) + 4arctan_2(\frac{1}{239})(|\ev{4}| + 13u'' + \left|\er{*}{4,arctan(\frac{1}{239})}\right|)}{16arctan_2(\frac{1}{5})-4arctan_2(\frac{1}{239})} + \left|\er{-}{16arctan_2(\frac{1}{5}),4arctan_2(\frac{1}{239})}\right|$\\
	
	Sea $u''' =$ máx$\left\{ u',\;u''\;|\ev{16}|,\;\left|\er{*}{16,arctan(\frac{1}{5})}\right|,\;|\ev{4}|,\;\left|\er{*}{4,arctan(\frac{1}{239})}\right|,\;\left|\er{-}{16arctan_2(\frac{1}{5}),4arctan_2(\frac{1}{239})}\right|\right\}$\\
	
	\pa
	
	$\Rightarrow \left|\ev{\pi}\right| \leq \frac{16arctan_2(\frac{1}{5})(u''' + 10u''' + u''') + 4arctan_2(\frac{1}{239})(u''' + 13u''' + u''')}{16arctan_2(\frac{1}{5})-4arctan_2(\frac{1}{239})} + u''' =$\\
	
	$(\frac{192arctan_2(\frac{1}{5}) + 60arctan_2(\frac{1}{239})}{16arctan_2(\frac{1}{5})-4arctan_2(\frac{1}{239})} + 1) u''' \leq 14u'''$\\
	
	Reemplazando $u'''=2^{1-t}$ nos queda que el error relativo del algoritmo de Machin:
	
	$\left|\ev{\pi}\right| \leq 14*2^{1-t}$
	
	
\end{subsection}

		\newpage
		\begin{subsection}{Serie de Ramanujan}
	\begin{equation*}
		\frac{1}{\pi} = \frac{\sqrt{8}}{9801} \sum_{n=0}^{\infty} \frac{(4n)! \, (1103 + 26390 n)}{(n!)^4 \, 396^{4n}}
    \end{equation*}
	
	\pa
	
	Definimos primero el error de un término de la sumatoria de la serie de Ramanujan.\\

	Dado $n \in \N$, sea $x=(4n)!(1103+26390n)$ e $y=(n!)^4396^{4n}$.\\
	
	$\e{div}{x,y,:} = \ev{x} - \ev{y} + \er{:}{x,y}$ (1).\\
	
	Como calculamos $x$ e $y$ en punto flotante tenemos que calcular el error cometido.\\
	
	\underline{Analicemos el error de $x=(4n)!(1103+26390n)$:}\\
	
	Llamamos $w=(4n)!$.\\

	$\ev{x} = \e{mult}{w,1103+26390n,*} = \ev{w}+\ev{1103+26390n}+\er{*}{w,1103+26390n}$\\
	
	Resta calcular el error de $w$ porque para $1103+26390n$ consideramos sólo el error de representación ya que se calcula en enteros (exacto, no hay $overflow$ ($n \leq 2$)).\\
	
	$\Rightarrow \left|\ev{x}\right| \leq \left|\ev{w}\right| + 2u$\\
	
	\underline{Analicemos el error de $y=(n!)^4396^{4n}$:}\\
	
	Llamamos $v=(n!)^4$ e $z=396^{4n}$.\\
	
	$\ev{y} = \e{mult}{v,z,*} = {\ev{v}+\ev{z}+\er{*}{v,z}}$\\
	
	donde,
	
	$\ev{v} = \e{pot}{n!,4,POT} = 4\ev{n!} + ln(n!) 4\ev{4} + \er{POT}{n!,4}$
	
	$\ev{z} = \e{pot}{396,4n,POT} = 4n\ev{396} + ln(396) 4n\ev{4n} + \er{POT}{396,4n}$\\
	
	Como $\ev{4n} = \e{mult}{4,n,*} = \ev{4} + \ev{n} + \er{*}{4,n}$\\
	
	$\Rightarrow \ev{z} = 4n\ev{396} + ln(396)4n(\ev{4} + \ev{n} + \er{*}{4,n}) + \er{POT}{396,4n}$\\
	
	Teniendo toda esta información podemos decir que el error de $y$ es:\\
	
	$4\ev{n!} + ln(n!) 4\ev{4} + \er{POT}{n!,4} + 4n\ev{396} + ln(396)4n(\ev{4} + \ev{n} + \er{*}{4,n}) + \er{POT}{396,4n} + \er{*}{v,z}$\\
	
	$\Rightarrow_{usando\;que\;ln(396)<6} \left|\ev{y}\right| \leq 4\left|\ev{n!}\right| + ln(n!)4u + 76nu + 3u$\\
	
	Finalmente, el error de un término de la sumatoria está acotado por:
	
	\pa

	$\left|\ev{\frac{x}{y}}\right| \leq \left|\ev{x}\right| + \left|\ev{y}\right| + \left|\er{:}{x,y}\right| \leq \left|\ev{(4n)!}\right| + 4\left|\ev{n!}\right| + ln(n!)4u + 76nu + 6u$\\
	
	Teniendo esta información, queremos realizar el análisis teórico para los primeros tres términos de la sumatoria. Para poder realizarlo, llamamos $a$ $(n=0)$ al primer término,
	$b$ $(n=1)$ al segundo y $c$ $(n=2)$ al tercero.
	
	Dado que tanto $a$ como $c$ corresponden a $n$ par son términos tal que $(-1)^n = 1$ en contraste con $b$ que tiene signo negativo. Por este motivo,\\
	tenemos que la arcotangente hasta el tercer término es: $a - b + c$.\\
	
	Analizamos primero el error cometido de realizar $a+c$.\\
	
	$\e{sum}{a,c,+} = \frac{a}{a+c}\ev{a} + \frac{c}{a+c}\ev{c} + \er{+}{a,c}$\\
	
	Analizamos ahora el error de restar el segundo término ($b$).\\
		
	$\e{restar}{a+c,b,-} = \frac{a+c}{a+c-b}\er{+}{a,c} - \frac{b}{a+c-b}\ev{b} + \er{-}{a+c,b} =$\\
	
	$\frac{a+c}{a+c-b}(\frac{a}{a+c}\ev{a} + \frac{c}{a+c}\ev{c} + \er{+}{a,c}) - \frac{b}{a+c-b}\ev{b} + \er{-}{a+c,b} =$\\
	
	$\frac{a\ev{a} + c\ev{c} - b\ev{b} + (a+c)\er{+}{a,c}}{a-b+c} + \er{-}{a+c,b} = \ev{a+c-b} = \ev{sum_2}$\\
	
	$\left|\ev{sum_2}\right| \leq \frac{|a||\ev{a}| + |c||\ev{c}| + |b||\ev{b}| + |a+c|\left|\er{+}{a,c}\right|}{|a-b+c|} + \left|\er{-}{a+c,b}\right|$\\
	
	Dado que $a=1103$, $b=\frac{659832}{396^4}$ y $c=\frac{8!*53883}{2^4*396^8}$.\\
	
	Basandonos en nuestra implementación y usando el hecho de que calculamos el error sólo para los $tres$ primeros términos de la sumatoria tenemos:\\
	
	El error de $a$ es el error de represetar $1103$ ya que lo adicionamos al final sin calcularlo.\\
	
	$\left|\ev{b}\right| \leq \left|\ev{4!}\right| + 4\left|\ev{1!}\right| + 82u = $\\
	
	$\left|\ev{1} + \ev{2} + \er{*}{1,2} + \ev{3} + \er{*}{1*2,3} + \ev{4} + \er{*}{1*2*3,4} + \ev{1} + \er{*}{1*2*3*4,1}\right| + 4\left|\e{mult}{1,1}\right| + 82u \leq \left|\ev{2}\right| + \left|\er{*}{1,2}\right| + \left|\ev{3}\right| + \left|\er{*}{1*2,3}\right| + \left|\ev{4}\right| + \left|\er{*}{1*2*3,4}\right| + \left|\er{*}{1*2*3*4,1}\right| + 4\left|\er{*}{1,1}\right| + 82u \leq 93u$\\
	
	$\left|\ev{c}\right| \leq \left|\ev{8!}\right| + 4\left|\ev{2!}\right| + 161u =$\\
	
	$\leq \left|\ev{5} + \ev{6} + \er{*}{5,6} + \ev{7} + \er{*}{5*6,7} + \ev{8} + \er{*}{5*6*7,8} + \ev{24} + \er{*}{5*6*7*8,24}\right| + 4\left|\e{mult}{1,2}\right| + 161u \leq \left|\ev{5}\right| + \left|\ev{6}\right| + \left|\er{*}{5,6}\right| + \left|\ev{7}\right| + \left|\er{*}{5*6,7}\right| + \left|\ev{8}\right| + \left|\er{*}{5*6*7,8}\right| + \left|\ev{24}\right| + \left|\er{*}{5*6*7*8,24}\right| + 4(\left|\ev{2}\right| + \left|\er{*}{1,2}\right|) + 161u \leq 178u$\\
	
	$\Rightarrow$\\
	$\left|\ev{sum_2}\right| \leq \frac{1103u + \frac{8!*53883}{2^4*396^8}178u + \frac{659832}{396^4}93u + (1103+\frac{8!*53883}{2^4*396^8})\left|\er{+}{1103,\frac{8!*53883}{2^4*396^8}}\right|}{1103 - \frac{659832}{396^4} + \frac{8!*53883}{2^4*396^8}} + \left|\er{-}{1103 + \frac{8!*53883}{2^4*396^8},\frac{659832}{396^4}}\right|$\\

	$\leq \frac{1103u + \frac{8!*53883}{2^4*396^8}178u + \frac{659832}{396^4}93u + (1103+\frac{8!*53883}{2^4*396^8})u}{1103 - \frac{659832}{396^4} + \frac{8!*53883}{2^4*396^8}} + u =$\\
	
	$\frac{2206u + \frac{8!*53883}{2^4*396^8}179u + \frac{659832}{396^4}93u}{1103 - \frac{659832}{396^4} + \frac{8!*53883}{2^4*396^8}} + u \leq 4u$\\
	
	\pa
	
	Por ultimo, $\pi_2=\frac{\frac{9801}{\sqrt{8}}}{sum_2}$\\
	
	Para calcular el error de $\pi_2$ calculamos el error de $\funcFull{h(k)}{\sqrt{k}}$.\\
	
	$\e{h}{k,\sqrt{ }} = \frac{\frac{1}{2\sqrt{k}}}{\sqrt{k}}k\ev{k} + \er{\sqrt{ }}{k} = \frac{1}{2\sqrt{k}}\frac{1}{\sqrt{k}}k\ev{k} + \er{\sqrt{ }}{k} = \frac{\ev{k}}{2} + \er{\sqrt{ }}{k}$\\
	
	$\ev{\frac{9801}{\sqrt{8}}} = \ev{9801} - \ev{\sqrt{8}} + \er{:}{9801,\sqrt{8}}$\\
	$\Rightarrow \left|\ev{\frac{9801}{\sqrt{8}}}\right| = \left|\ev{9801}\right| + \left|\frac{\ev{8}}{2}\right| + \left|\er{\sqrt{ }}{8}\right| + \left|\er{:}{9801,\sqrt{8}}\right| \leq \frac{7}{2}u$\\
		
	$\ev{\pi_2} = \e{div}{\frac{9801}{\sqrt{8}},sum_2} = \ev{\frac{9801}{\sqrt{8}}} - \ev{sum_2} + \er{:}{\frac{9801}{\sqrt{8}},sum_2} \leq \left|\ev{\frac{9801}{\sqrt{8}}}\right| + \left|\ev{sum_2}\right| + \left|\er{:}{\frac{9801}{\sqrt{8}},sum_2}\right| \leq \frac{7}{2}u + 5u = \frac{17}{2}u$\\
		
\end{subsection}
		
	\end{subsection}
	\begin{subsection}{Implementación}
		Al necesitar manejar presición arbitraria los tipos de datos nativos del lenguaje C++ no satisfacían nuestras necesidades.
		
		Una solución encontrada a este problema, fue implementar un tipo de datos (clase $Real$) con esta funcionalidad (presición variable). La clase $Real$ implementa los operadores básicos: suma, resta, multiplicación. división y asignación, los observadores sobre la presición y truncamiento, y funciones para visualización de la información del $Real$ en stdout. Se exporta además dos métodos con funcionalidad importante, uno de ellos para convertir el $Real$ en $double$ (tipo nativo de C++) y otro método que dado un $double$ modifica la instancia $Real$, representando ahora, el valor pasado por parámetro.
		
		Como precondición de este trabajo práctico la precisión máxima de digitos (en base 2) del cálculo de $\pi$ debe ser a lo sumo 51 bits.
		Utilizamos el tipo $double$ de C++ como cimiento de nuestra clase $Real$. Las operaciones aritméticas sobre $Real$ consisten en convertirlos en $double$, realizar la operación correspondiente y transformar el resultado nuevamente en un $Real$, aplicándole previamente el algoritmo de truncamiento
		 %o redondeo según corresponda 
		acorde a los parámetros de entrada del programa (cantidad de dígitos).
		
		Fuera de la clase implementamos el cálculo de raíz cuadrada, arco tangente y exponenciación de $Real$. Se tomó esta decisión porque el uso de estas operaciones sólo tiene sentido en instancias de $Real$ en situaciones complejas (por ejemplo el cálculo de $\pi$). Tomamos como modelo la implementación de estas operaciones en C++ que requiere la inclusión de una biblioteca (cmath).
		
		La implementación de todos los algoritmos del cálculo de $\pi$ maximizan el uso de variables de tipo entero ya que las cuentas en número flotante poseen error mientras que enteros no. Cuando los cálculos dejan de tener sentido en el mundo de los enteros o su precisión no es suficiente, se crea una instancia de $Real$ que represente al valor entero y se continúa operando sobre este tipo. Por ese motivo una instancia de $Real$ se puede generar sólo a partir de un entero de 64 bits con signo.\\
		
		\underline{Observación:} Dada la utilización de la clase $Real$ en este trabajo práctico no fue necesario crear una instancia de la misma a partir de un $double$ ya que por ejemplo si pretendemos crear un $Real$ que represente $\sqrt{8}$ se podría generar a partir de $8$ y luego aplicarle la función raíz. Este constructor seria de mayor necesidad si el diseño de esta clase tuviera como fin un uso desconocido.\\
		   
	   A continuación detallamos las optimizaciones de los algoritmos:

		\begin{itemize}
			\item \underline{Gregory:} Refactorizamos nuestra implementación original de la serie que consistía en respetar el orden de los sumandos según el enunciado como las operaciones internas de cada uno de ellos. A continuación se detalla en lenguaje matemático los cambios realizados.\\
							
				$\sum_{n=0}^{\infty}\frac{(-1)^n}{2n+1} = \sum_{n \in \mathbb{N} / n\;mod(2)=0}^{\infty}\frac{1}{2n+1}-\sum_{n \in \mathbb{N} / n\;mod(2)=1}^{\infty}\frac{1}{2n+1}$\\
				
				Al hacerlo de esta manera en la ejecución de cada iteración del algoritmo evitamos el cálculo de una potencia o la utilización de una sentencia $if$ para determinar el signo, al costo de almacenar dos números que acumulan las sumas.
				
				Por otro lado, decidimos calcular $2n+1$ en un tipo de dato entero de 64 bits sin signo ya que de esta forma podemos evitar el error en el cálculo (a diferencia de si se realizara con números flotantes) y sería necesario un $n$ muy grande (cantidad de sumandos de la serie) para producir $overflow$.
				
			\item \underline{Machin:} Este algoritmo ejecuta la función $arctan$ la cual realiza la optimización antes mencionada.
			
			\item \underline{Ramanujan:} Este algoritmo fue rediseñado en varias oportunidades al observar diferentes problemas en cada implementación que realizamos. Entre sus operaciones encontramos la función factorial la cual en pocas iteraciones (en enteros sin signo de 64 bits) produce $overflow$. Sabiendo que el rango de valores representable en $double$, y por consiguiente en $Real$ es mayor al de cualquier entero de 64 bits, optamos por realizar la operación factorial en nuestro tipo de datos ($Real$).
			
		Nuestra primer aproximación fue definir la función factorial, la cual dado un $n$ calculaba la $\prod_{k=1}^{n}{k}$. Observamos que la cantidad de operaciones elementales que factorial realiza en cada llamada es lineal en función de $n$. El uso que se le da en el algoritmo de $Ramanujan$ es calcular el factorial de números consecutivos pudiéndose entonces, reutilizar el resultado de la iteración previa.\VSP
		
		A continuación se muestra un breve pseudcódigo del uso de la función factorial en el algoritmo de $Ramanujan$ (primera implementación).
		
		$\func{Ramanujan}{n}$\\
		\tab\FOR j=0 \TO n\\
		\tab\tab$factorial(4*j)$\\
		\tab\tab\tab.\\
		\tab\tab\tab.\\
		\tab\tab\tab.\\
		\tab\END
		
		\VSP
		
		Sabiendo que factorial es \Ode{n} y la cantidad de iteraciones del ciclo $for$ es $n$ y suponiendo que el resto de las operaciones del ciclo son constantes el algoritmo tiene complejidad cuadrática en función de $n$.\VSP

		A continuación se muestra un breve pseudocódigo del cálculo del factorial acumulando en una variable el resultado de la iteración anterior.\VSP
		
		$\func{Ramanujan}{n}$\\
		\tab $acum\_fact=1$\\
		\tab\FOR j=1 \TO n\\
		\tab\tab $i = 4*j$\\
		\tab\tab$acum\_fact = acum\_fact*(i-1)*(i-2)*(i-3)*i$\\
		\tab\tab\tab.\\
		\tab\tab\tab.\\
		\tab\tab\tab.\\
		\tab\END
		
		\VSP
		
		De esta forma, se realiza una cantidad constante de multiplicaciones en cada iteración del ciclo $for$ para el cálculo del factorial (suponiendo el resto de las operaciones del ciclo de costo constante) el algoritmo tiene complejidad lineal en función de $n$.
		
		La complejidad final del algoritmo de $Ramanujan$ no es esta ya que entre las operaciones del ciclo $for$ se encuentra la operación potencia con costo lineal en función del exponente ($4*j$). La complejidad es de \Ode{n^2}.
		
		\underline{Observación:} De manera similar a la implementación del factorial, se puede realizar el cálculo de la potencia utilizando acumuladores. Dado que esta función es utilizada en más de un algoritmo decidimos reutilizar código.
		
		Cabe mencionar que el primer término de la serie es un valor constante y entero, por lo que decidimos excluirlo del cálculo de la sumatoria y adicionarlo al final (por ese motivo la implementación final itera de 1 a n).
		\end{itemize}
		
		Para facilitar el desarrollo del programa se generaron varias funciones auxiliares de impresión de información. A la hora de hacer los test no era suficiente la visualización provista por la salida standard de C++. Estas visualizaciones como por ejemplo mostrar el $double$ bit a bit facilitaron la comprensión y analisis de errores programáticos.
		
	\end{subsection}
\end{section}

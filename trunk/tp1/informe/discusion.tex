\begin{section}{Discusión}
	En esta sección, buscaremos conclusiones a la información suministrada por los gráficos de la sección anterior.
	
	El primer gráfico presenta una primer aproximación al error relativo de los tres algoritmos (Figura:\ref{fig:51p}).
	Observamos que a medida que la cantidad de términos de las series aumenta, el error cometido en el cálculo de $\pi$ de cada una de ellas disminuye.
	
	El error introducido al aproximar $\pi$ con la serie de $Gregory$ es logarítmico (en la escala del gráfico) en función de la cantidad de términos de la serie calculados. Podemos realizar un cambio de escala en el eje $y$, transformándolo en lineal obtenemos que el error de la serie de $Greory$ decrece linealmente conforme aumenta la cantidad de iteraciones.
	
	En la primer aproximación a la comprensión del error relativo cometido al aproximar $\pi$ con la fórmula de $Machin$ en función de la cantidad de iteraciones, vemos que los valores obtenidos en cada iteración forman una recta con pendiente negativa. Como la escala utilizada para representar dicho error es la logarítmica, podemos decir que el error decrece exponencialmente.
	
	Además observamos que a partir de la décima iteración el gráfico no muestra de forma clara lo que ocurre con el error en la fórmula de $Machin$. Suponemos que se debe al hecho de que los valores que representan a $Machin$ se superponen con los valores correspondientes a $Ramanujan$.
	
	Por otro lado, podemos apreciar que con pocas iteraciones la serie de $Ramanujan$ obtiene un error constante, similar observación puede hacerse a $Machin$ a partir de la iteración diez.\\
	
	Generamos entonces un nuevo gráfico (Figura:\ref{fig:greg-ram}) para darle sustento a estas conclusiones, o refutarlas y buscar un nuevo enfoque sobre los resultados anteriores.

	Con este gráfico nos vemos tentados a validar lo supuesto en un principio, ya que se puede apreciar que entre estas iteraciones el error cometido por ambos métodos es el mismo. Más aún se podría concluir a partir del gráfico que el error es cero lo cual no es cierto. Al ser un error tan chico el gráfico no ayuda a la correcta interpretación de los datos. Sabemos que, a pesar de seguir iterando no se podria obtener una mejor aproximación a $\pi$ ya que el impedimento viene dado por la cantidad de digitos de presición utilizada.
	A partir de esto, deducimos que al tratarse de magnitudes tan pequeñas los gráficos son informativos sólo en rasgos generales y no en detalles.\\	

	Como tercer resultado, tenemos al gráfico que muestra el error relativo con respecto a la cantidad de dígitos, fijando la cantidad de iteraciones en $42$ (Figura:\ref{fig:42it})
	
	La apreciación (lineal/exponencial) del decrecimiento de los errores en función de la cantidad de dígitos es la misma que en función de la cantidad de iteraciones para $Ramanujan$ y $Machin$. El primero de los dos en este caso, decrece lineal (pensándolo en escala logarítmica) sin cambiar el valor de su pendiente a partir de cierto valor. Parece reducir entonces, de manera exponencial el error cometido conforme aumenta la presición utilizada.
	
	$Gregory$ muestra una pendiente logarítmica monótona decreciente hasta un $t$ cercano a doce, donde según este gráfico, se estabiliza y se transforma en constante. Si esto sucede, significaría que no tiene sentido seguir calculando más iteraciones, ya que el error de estos dígitos no van a mejorar con respecto al valor exacto de la constante a calcular, es decir, si al variar la cantidad de bits el error no disminuye, entonces existe al menos un bit con diferencia (entre $Gregory$ y el exacto), la cual nunca será cero (la diferencia, vala la redundancia). Por lo tanto, podríamos suponer que la serie converge a $\pi$ más el error relativo, lo que implicaría que Gregory en efecto no calcula dicha constante. Esta serie de pasos lógicos, nos hace llevar a la conclusión que la escala del gráfico nos hace perder información sensible y crucial para el entendimiento, ya que en caso contrario, estaríamos ante la presencia de un absurdo.
	
	Observamos al igual que en el gráfico anterior que el error cometido por $Machin$ con 51 dígitos de presición no aparece graficado mientras que con el resto de las precisiones si. Concluimos a partir de esto, que a pesar que el error cometido por ambas series ($Machin$ y $Ramanujan$) es similar con 42 iteraciones, se necesita la máxima presicion brindada para que se haga totalmente despreciable la diferencia, si existe.\\
	
	El último gráfico adjuntado, Figura:\ref{fig:5it}, plantea fijar la cantidad de iteraciones en un valor menor (cinco en este caso), y analizar si existe diferencias con respecto al anterior de 42.
	
	Vemos que con poca presición los errores siguen siendo similares.
	
	$Machin$ consigue mejores aproximaciones que $Gregory$ fijando una cantidad de iteraciones, cuando la presición es baja pierde mayor información al truncar lo que se ve reflejado en la diferencia de errores al aumentar dicha presición (necesita más dígitos para estabilizarse).
	
	En el primer y segundo algoritmo, $Gregory$ y $Machin$, podemos inferir que existe una relación entre la cantidad de iteraciones de la serie la precisión utilizada, ahora al tener una cantidad inferior de iteraciones, el hecho que exista la posibilidad de calcular con precisión mayor a 30 bits (para $Machin$ por ejemplo) no aporta. Esto en cambio, no sucede tan notoriamente con $Ramanujan$. Si bien los tres algoritmos necesitan de realizar la sumatoria infinita para converger a la constante, dado un valor fijo para el cálculo de la sumatoria y/o bits fijos, su mejor aproximación vendrá dada por $Ramanujan$, seguido de $Machin$ y luego $Gregory$.
	\\

\end{section}

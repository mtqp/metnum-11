\begin{subsection}{Fórmula de Machin}

	\begin{equation*}
		\frac{\pi}{4} = 4 \; \mathrm{arctan}(1/5) - \mathrm{arctan}(1/239), \text{con } \mathrm{arctan}(x) = \sum_{n=0}^{\infty} \left(-1\right)^n \frac{x^{2n+1}}{2n+1} \quad \text{cuando } \left|x\right|<1
    \end{equation*}
	
	Definimos primero el error de calcular $arctan(k)$ con $|k|<1$. Para ello definimos el error de cada uno de los términos de la sumatoria.\\
	
	Dado $n \in \N$, sea $x=k^{2n+1}$ e $y=2n+1$.\\
	
	$\e{div}{x,y,:} = \ev{x} - \ev{y} + \er{:}{x,y}$ como vimos previamente.\\
	
	Como $y$ es de tipo entero (exacto) en este caso con $n \leq 2$ lo que implica $y \leq 5$ no hay $overflow$. Resta calcular el error de $x$.\\
	
	\underline{Analicemos el error de $x=k^y$ ($k=\frac{1}{5}$ o $k=\frac{1}{239} \Rightarrow k>0$):}\\
	
	$\funcFull{$\ev{x}$}{\frac{yk^{y-1}k}{k^y}\ev{k}+\frac{k^y ln(k) y}{k^y}\ev{y} + \er{POT}{k,y}} = y\ev{k} + ln(k) y\ev{y} + \er{POT}{k,y}$ (2)\\ 
	
	$\Rightarrow_{Reemplazando \; (2) \; en \; (1)}$\\
	
	Tenemos entonces que el error de un término de la arcotangente es:\\
	
	$y\ev{k} + ln(k) y\ev{y} + \er{POT}{k,y} - \ev{y} + \er{:}{x,y}$\\
	
	$\Rightarrow  \left|\ev{t_i}\right| = \left|\ev{\frac{k^y}{y}}\right| \leq \left|y\ev{k}\right| + \left|ln(k)y\ev{y}\right| + \left|\er{POT}{k,y}\right| + \left|\ev{y}\right| + \left|\er{:}{x,y}\right| =$\\
	
	$(2n+1)(\left|\ev{k}\right| + \left|ln(k)\right|\left|\ev{y}\right|) + \left|\er{POT}{k,(2n+1)}\right| + \left|\ev{(2n+1)}\right| + \left|\er{:}{k^{2n+1},2n+1}\right|$\\
	
	Sea $u$ el error máximo de todas operaciones en módulo.\\
	
	$\Rightarrow \left|\ev{\frac{k^{2n+1}}{2n+1}}\right| \leq (2n+1)(u + \left|ln(k)\right|u) + 3u$\\
	
	Teniendo esta información, queremos realizar el análisis teórico para los primeros tres términos de la sumatoria. Para poder realizarlo, llamamos $a$ $(n=0)$ al primer término,
	$b$ $(n=1)$ al segundo y $c$ $(n=2)$ al tercero.
	
	Dado que tanto $a$ como $c$ corresponden a $n$ par son términos tal que $(-1)^n = 1$ en contraste con $b$ que tiene signo negativo. Por este motivo,\\
	tenemos que la arcotangente hasta el tercer término es: $a - b + c$.\\
	
	Analizamos primero el error cometido de realizar $a+c$.\\
	
	$\e{sum}{a,c,+} = \frac{a}{a+c}\ev{a} + \frac{c}{a+c}\ev{c} + \er{+}{a,c}$\\
	
	Analizamos ahora el error de restar el segundo término ($b$).\\
		
	$\e{restar}{a+c,b,-} = \frac{a+c}{a+c-b}\er{+}{a,c} - \frac{b}{a+c-b}\ev{b} + \er{-}{a+c,b} =$\\
	
	$\frac{a+c}{a+c-b}(\frac{a}{a+c}\ev{a} + \frac{c}{a+c}\ev{c} + \er{+}{a,c}) - \frac{b}{a+c-b}\ev{b} + \er{-}{a+c,b} =$\\
	
	$\frac{a\ev{a} + c\ev{c} - b\ev{b} + (a+c)\er{+}{a,c}}{a-b+c} + \er{-}{a+c,b} = \ev{a+c-b} = \ev{arctan_2}$\\
		
	Para la fórmula de Machin se necesita calcular $arctan(\frac{1}{5})$ y $arctan(\frac{1}{239})$.\\
	
	$\e{arctan_2}{\frac{1}{5}} = \frac{\frac{1}{5}\ev{a} + (\frac{1}{5})^6\ev{c} - \frac{1}{3}(\frac{1}{5})^3\ev{b} + (\frac{1}{5} + (\frac{1}{5})^5)\er{+}{\frac{1}{5},(\frac{1}{5})^5}}{\frac{1}{5} + (\frac{1}{5})^6 - \frac{1}{3}(\frac{1}{5})^3} + \er{-}{\frac{1}{5} + \frac{1}{5} + (\frac{1}{5})^5,(\frac{1}{5})^3}$\\
	
	$\Rightarrow \left| \e{arctan_2}{\frac{1}{5}} \right| \leq \frac{\frac{1}{5}\left| \ev{a} \right| + (\frac{1}{5})^6\left| \ev{c} \right| + \frac{1}{3}(\frac{1}{5})^3\left| \ev{b} \right| + (\frac{1}{5} + (\frac{1}{5})^5)\left| \er{+}{\frac{1}{5},(\frac{1}{5})^5} \right|}{\frac{1}{5} + (\frac{1}{5})^6 - \frac{1}{3}(\frac{1}{5})^3} + \left|\er{-}{\frac{1}{5} + \frac{1}{5} + (\frac{1}{5})^5,(\frac{1}{5})^3} \right|$\\
	
	Como vimos antes:\\
	
	$\left|\ev{a}\right| \leq (u + \left|ln(\frac{1}{5})\right|u) + 3u \leq 6u$\\
	
	$\left|\ev{b}\right| \leq 3(u + \left|ln(\frac{1}{5})\right|u) + 3u \leq 12u$\\
	
	$\left|\ev{c}\right| \leq 5(u + \left|ln(\frac{1}{5})\right|u) + 3u \leq 18u$\\
	
	$\Rightarrow \left| \e{arctan_2}{\frac{1}{5}} \right| \leq \frac{\frac{6}{5}u + (\frac{1}{5})^618u + (\frac{1}{5})^34u + (\frac{1}{5} + (\frac{1}{5})^5)\left| \er{+}{\frac{1}{5},(\frac{1}{5})^5} \right|}{\frac{1}{5} + (\frac{1}{5})^6 - \frac{1}{3}(\frac{1}{5})^3} + \left|\er{-}{\frac{1}{5} + \frac{1}{5} + (\frac{1}{5})^5,(\frac{1}{5})^3} \right|$\\
		
	Si bien no conocemos cuál es el error exacto que posee cada operación, podemos tomar $u'$ como el error máximo de todas ellas en módulo, tal que $u' \geq u$. Por lo tanto el resultado obtenido es menor igual a:\\
	
	$\frac{\frac{6}{5}u' + (\frac{1}{5})^618u' + (\frac{1}{5})^34u' + (\frac{1}{5} + (\frac{1}{5})^5)u'}{\frac{1}{5} + (\frac{1}{5})^6 - \frac{1}{3}(\frac{1}{5})^3} + u'$\\
	
	Finalmente calculamos el error de la fórmula de Machin.\\
	
	Sea $\funcFull{f'}{z*4}$,\\
	
	$\e{f'}{z,4,*} = \ev{z} + \ev{4} + \er{*}{z,4}$ (3)\\
	
	Por otro lado calculamos el error de $z$ instanciando $g'$ con $k=4*arctan(\frac{1}{5})$ y $b=arctan(\frac{1}{239})$\\
	
	$\ev{z} = \frac{4*arctan(\frac{1}{5})}{4*arctan(\frac{1}{5})-arctan(\frac{1}{239})}\ev{4*arctan(\frac{1}{5})} - \frac{arctan(\frac{1}{239})}{4*arctan(\frac{1}{5})-arctan(\frac{1}{239})}\ev{arctan(\frac{1}{239})} + \er{-}{\frac{1}{5},\frac{1}{239}} (4)$
	
	$\Rightarrow_{Reemplazando \; (4) \; en \; (3)}$\\
	
	$\frac{4*arctan(\frac{1}{5})}{4*arctan(\frac{1}{5})-arctan(\frac{1}{239})}\ev{4*arctan(\frac{1}{5})} - \frac{arctan(\frac{1}{239})}{4*arctan(\frac{1}{5})-arctan(\frac{1}{239})}\ev{arctan(\frac{1}{239})} + \er{-}{\frac{1}{5}),\frac{1}{239})} + \ev{4} + \er{*}{4*arctan(\frac{1}{5}-arctan(\frac{1}{239})),4}$\\
	
	$\Rightarrow$\\
	
	$\frac{4*arctan(\frac{1}{5})}{4*arctan(\frac{1}{5})-arctan(\frac{1}{239})}(\ev{arctan(\frac{1}{5})}+\ev{4}+\er{*}{arctan(\frac{1}{5}),4}) - \frac{arctan(\frac{1}{239})}{4*arctan(\frac{1}{5})-arctan(\frac{1}{239})}\ev{arctan(\frac{1}{239})} + \er{-}{\frac{1}{5}),\frac{1}{239})} + \ev{4} + \er{*}{z,4} < $
	
	$\Rightarrow_{Acotando \; los \; errores \; por \; k, \; es \; menor \; a:}$\\
	
	$< \frac{4*arctan(\frac{1}{5})}{4*arctan(\frac{1}{5})-arctan(\frac{1}{239})}3k - \frac{arctan(\frac{1}{239})}{4*arctan(\frac{1}{5})-arctan(\frac{1}{239})}k+3k $
	
	$\Rightarrow$\\
	
	$< \frac{4*arctan(\frac{1}{5})}{4*arctan(\frac{1}{5})-arctan(\frac{1}{239})}3k + \frac{arctan(\frac{1}{239})}{4*arctan(\frac{1}{5})-arctan(\frac{1}{239})}3k+3k $
	
	$\Rightarrow$\\
	
	$< (\frac{4*arctan(\frac{1}{5})+arctan(\frac{1}{239})}{4*arctan(\frac{1}{5})-arctan(\frac{1}{239})} + 1) 3k <  (\frac{5*arctan(\frac{1}{5})}{3*arctan(\frac{1}{239})} + 1)3k < 240k$\\
		
	Reemplazando $k=2^{1-t}$ nos queda que el error relativo del algoritmo de Machin:
	
		$\ev{\pi} < 240*2^{1-t}$
	
	
\end{subsection}

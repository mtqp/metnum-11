\begin{subsubsection}{Serie de Gregory}
		\begin{equation*}
		\frac{\pi}{4} = \sum_{n=0}^{\infty} \frac{\left(-1\right)^n}{2n+1}
		\end{equation*}

	\pa
	
	Definimos primero el error de un término de la serie.\\
		
	Dado $n \in \N$, sea $x=1$ e $y=2n+1$.\\
		
	$\e{div}{x,y,:} = \ev{x} - \ev{y} + \er{:}{x,y}$ como vimos previamente.\\
	
	Este error corresponde al error de un término de la serie de Gregory porque $x$ e $y$ son de tipo entero (exactos), en este caso con $n \leq 2$ lo que implica $y \leq 5$ ($x$ constante) no hay $overflow$.\\
	
	%\underline{Nota:} El error de representación de $x$ e $y$ (al convertirlos a punto flotante para operar) será considerado en el cálculo de error de $\func{f}{x,y}$.\\
	
	Teniendo esta información, llamamos $a$ $(n=0)$ al primer término de la serie, $b$ $(n=1)$ al segundo y $c$ $(n=2)$ al tercero.
	
	Dado que tanto $a$ como $c$ corresponden a $n$ par son términos tal que $(-1)^n = 1$ en contraste con $b$ que tiene signo negativo. Por este motivo,\\
	tenemos que la serie de \texttt{Gregory} hasta el tercer término es: $a - b + c$.\\
	
	Analizamos primero el error cometido de realizar $a+c$.\\
	
	$\e{sum}{a,c,+} = \frac{a}{a+c}\ev{a} + \frac{c}{a+c}\ev{c} + \er{+}{a,c}$\\
	
	Analizamos ahora el error de restar el segundo término ($b$).\\
		
	$\e{restar}{a+c,b,-} = \frac{a+c}{a+c-b}\er{+}{a,c} - \frac{b}{a+c-b}\ev{b} + \er{-}{a+c,b} =$\\
	
	$\frac{a+c}{a+c-b}(\frac{a}{a+c}\ev{a} + \frac{c}{a+c}\ev{c} + \er{+}{a,c}) - \frac{b}{a+c-b}\ev{b} + \er{-}{a+c,b} =$\\
	
	$\frac{a\ev{a} + c\ev{c} - b\ev{b} + (a+c)\er{+}{a,c}}{a-b+c} + \er{-}{a+c,b} = \ev{a+c-b} = \ev{\frac{\pi}{4}_2}$\\
	
	Entonces el error en el tercer término ($\ev{\pi_2}$) es el siguiente:\\
	
	$\e{mult}{4,a+c-b} = \ev{4} + \ev{a+c-b} + \er{*}{4,a+c-b} = \ev{4} + \frac{a\ev{a} + c\ev{c} - b\ev{b} + (a+c)\er{+}{a,c}}{a-b+c} + \er{-}{a+c,b} + \er{*}{4,a+c-b}$\\
	
	Dado que $a=1$, $b=\frac{1}{3}$ y $c=\frac{1}{5}$. Reemplazando por los términos\\
	correspondientes obtenemos:\\
	
	$\ev{4} + \frac{\ev{1} - \ev{1} + \er{:}{1,1} - (\frac{1}{3})(\ev{1} - \ev{3} + \er{:}{1,3}) + (\frac{1}{5})(\ev{1} - \ev{5} + \er{:}{1,5}) + \frac{6}{5} \er{+}{1,\frac{1}{5}}}{\frac{13}{15}} + \er{-}{\frac{6}{5},\frac{1}{3}} + \er{*}{4,\frac{13}{15}}$\\
	
	Como $\ev{1}=0$ ya que al tener un bit implícito no es necesario que la precisión sea mayor a $uno$ $\Rightarrow$\\
	
	$\ev{4} + \frac{\er{:}{1,1} - (\frac{1}{3})(- \ev{3} + \er{:}{1,3}) + (\frac{1}{5})(- \ev{5} + \er{:}{1,5}) + \frac{6}{5} \er{+}{1,\frac{1}{5}}}{\frac{13}{15}} + \er{-}{\frac{6}{5},\frac{1}{3}} + \er{*}{4,\frac{13}{15}}$\\
	
	$\Rightarrow \ev{\pi_2} = \left| \ev{4} + \frac{15\er{:}{1,1} + 5\ev{3} - 5\er{:}{1,3}) - 3\ev{5} + 3\er{:}{1,5}) + 18\er{+}{1,\frac{1}{5}}}{13} + \er{-}{\frac{6}{5},\frac{1}{3}} + \er{*}{4,\frac{13}{15}} \right|$\\
	
	$\leq \left|\ev{4}\right| + \frac{ \left|15\er{:}{1,1}\right| + \left|5\ev{3}\right| + \left|5\er{:}{1,3})\right| + \left|3\ev{5}\right| + \left|3\er{:}{1,5})\right| + \left|18\er{+}{1,\frac{1}{5}}\right|}{13} + \left|\er{-}{\frac{6}{5} ,\frac{1}{3}}\right| + \left|\er{*}{4,\frac{13}{15}}\right|$\\
		
	Si bien no conocemos cuál es el error exacto que posee cada operación, podemos acotarlas por el error máximo de todas ellas en módulo, llamémoslo $u$. Por lo tanto el resultado obtenido es menor igual a:\\

	$u + \frac{15u + 5u + 5u + 3u + 3u + 18u}{13} + u + u = \frac{49}{13}u + 3u = \frac{88}{13}u < 7u$\\

	Sabiendo que la implementación de estos algoritmos será realizada en una máquina con arítmetica binarias, podemos inferir que el error en los cálculos se producirá a partir de cierto bit. El error entonces depende de la conversión de los valores enteros a números con coma, así como también la presición que estemos utilizando. Tomamos de esta forma a $u=2^{1-t}$.
	
	Acotando el error de Gregory, tenemos que $\ev{\pi_2} < 7*2^{1-t}$\\
	
	\underline{Nota:} Este último resultado (el valor de $u$) aplica tanto al análisis previamente realizado, como a los dos subsiguientes (algoritmos de Machin y Ramanujan). Para simplificar los cálculos, y disminuir la probabilidad de arrastrar errores de cuentas, se acotará por $u$ el que luego será reemplazado por $2^{1-t}$.
\end{subsubsection}
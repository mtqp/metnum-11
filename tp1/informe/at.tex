\documentclass[12pt,titlepage]{article}
\usepackage[spanish]{babel}
\usepackage[utf8]{inputenc}
\usepackage{amsfonts}
\usepackage{amsmath}
\usepackage{amssymb}
\usepackage{color}
\usepackage{graphicx} % para insertar imagenes
\usepackage{verbatim}

\newcommand{\func}[2]{\texttt{#1}(#2)}
\newcommand{\funcFull}[2]{\texttt{#1}=#2}
\newcommand{\tab}{\hspace*{2em}}
\newcommand{\FOR}{\textbf{for }}
\newcommand{\TO}{\textbf{ to }}
\newcommand{\IF}{\textbf{if }}
\newcommand{\WHILE}{\textbf{while }}
\newcommand{\THEN}{\textbf{then }}
\newcommand{\ELSE}{\textbf{else }}
\newcommand{\RET}{\textbf{return }}
\newcommand{\MOD}{\textbf{ \% }}
\newcommand{\OR}{\textbf{ or }}
\newcommand{\AND}{\textbf{ and }}
\newcommand{\tOde}[1]{\tab \small{O($#1$)}}
\newcommand{\Ode}[1]{O($#1$)}
\newcommand{\Thetade}[1]{{\small$\Theta$($#1$)}}
\newcommand{\Omegade}[1]{{\small$\Omega$($#1$)}}
\newcommand{\VSP}{\vspace*{3em}}
\newcommand{\pa}{\vspace{5mm}}
\newcommand{\e}[2]{\varepsilon  _{#1}({#2})}
\newcommand{\er}[2]{\varepsilon _{({#1})}^{({#2})}}
\newcommand{\ev}[1]{\varepsilon _{#1}}
\newcommand{\N}{\mathbb{N}}
\newenvironment{pseudo}{\begin{tabular}{p{11cm}l}}{\end{tabular}\VSP}

\newcommand{\gra}[1]{{\noindent\centering\includegraphics[width=14cm]{#1}}\\}

\begin{document}
	\begin{section}{Serie de Gregory}

		\large
		
		Definamos primero el error de un término de la serie de Gregory.\\

		Dado $n \in \N$, sea $\funcFull{f(x,y)} {\frac{x}{y}}$, donde $x=(-1)^n$ e $y=2n+1$.\\
		
		El error de $\func{f}{x,y}$ es el error de un término de la serie de Gregory ya que $x$ e $y$ son de tipo entero (exactos).\\
		
		\underline{Analicemos el error de $\func{f}{x,y}$:}\\
		
		$\funcFull{$\e{f}{x,y,:}$}{\frac{\frac{x}{y}\ev{x}}{f} + \frac{\frac{-xy}{y^2}\ev{y}}{f} + \er{:}{x,y}} = 
		\frac{\frac{x}{y}\ev{x}}{\frac{x}{y}} + \frac{\frac{-x}{y}\ev{y}}{\frac{x}{y}} + \er{:}{x,y} =$\\
		
		$\ev{x} - \ev{y} + \er{:}{x,y}$\\
		
		Teniendo esta información, queremos realizar el análisis teórico sobre la serie de Gregory para los primeros tres términos. Para poder realizarlo, llamamos $a$ al primer término de la serie,
		$b$ al segundo y $c$ al tercero (llamamos $k=a+b$).\\
		
		Sea $\funcFull{g(k,c)}{k+c}$\\
		
		$\e{g}{k,c,+} = \frac{k}{k+c}\ev{k} + \frac{c}{k+c}\ev{c} + \er{+}{k,c}$\\
		
		$\e{g}{a+b,c,+} = \frac{a+b}{a+b+c}(\frac{a}{a+b}\ev{a} + \frac{b}{a+b}\ev{b} + \er{+}{a,b}) + \frac{c}{a+b+c}\ev{c} + \er{+}{a+b,c} =$\\
		
		$\frac{a\ev{a} + b\ev{b} + c\ev{c} + (a+b)\er{+}{a,b}}{a+b+c} + \er{+}{a+b,c}$\\
		
		Dado que $a=1$, $b=\frac{-1}{3}$ y $c=\frac{1}{5}$. Reemplazando por los términos correspondientes obtenemos:\\
		
		$\frac{\ev{1} - \ev{1} + \er{:}{1,1} + (\frac{-1}{3})(\ev{-1} - \ev{3} + \er{:}{-1,3}) + (\frac{1}{5})(\ev{1} - \ev{5} + \er{:}{1,5}) + \frac{2}{3} \er{+}{1,\frac{-1}{3}}}{\frac{13}{15}} + \er{+}{\frac{2}{3},\frac{1}{5}}$\\
		
		\color{red}{Falta calcular el error de multiplicar esto por 4}
	\end{section}
\end{document}

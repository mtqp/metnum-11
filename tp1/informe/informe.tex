\documentclass[12pt,titlepage]{article}
\usepackage[spanish]{babel}
\usepackage[utf8]{inputenc}
\usepackage{amsfonts}
\usepackage{amsmath}
\usepackage{amssymb}
\usepackage{color}
\usepackage{graphicx} % para insertar imagenes
\usepackage{caratulaMetNum}
\usepackage{verbatim}

\newcommand{\func}[2]{\texttt{#1}(#2)}
\newcommand{\funcFull}[2]{\texttt{#1}=#2}
\newcommand{\tab}{\hspace*{2em}}
\newcommand{\FOR}{\textbf{for }}
\newcommand{\TO}{\textbf{ to }}
\newcommand{\IF}{\textbf{if }}
\newcommand{\WHILE}{\textbf{while }}
\newcommand{\THEN}{\textbf{then }}
\newcommand{\ELSE}{\textbf{else }}
\newcommand{\RET}{\textbf{return }}
\newcommand{\MOD}{\textbf{ \% }}
\newcommand{\OR}{\textbf{ or }}
\newcommand{\AND}{\textbf{ and }}
\newcommand{\tOde}[1]{\tab \small{O($#1$)}}
\newcommand{\Ode}[1]{O($#1$)}
\newcommand{\Thetade}[1]{{\small$\Theta$($#1$)}}
\newcommand{\Omegade}[1]{{\small$\Omega$($#1$)}}
\newcommand{\VSP}{\vspace*{3em}}
\newcommand{\pa}{\vspace{5mm}}
\newcommand{\e}[2]{\varepsilon  _{#1}({#2})}
\newcommand{\er}[2]{\varepsilon _{({#1})}^{({#2})}}
\newcommand{\ev}[1]{\varepsilon _{#1}}
\newcommand{\N}{\mathbb{N}}
\newenvironment{pseudo}{\begin{tabular}{p{11cm}l}}{\end{tabular}\VSP}

\newcommand{\gra}[1]{{\noindent\centering\includegraphics[width=14cm]{#1}}\\}

%\title{{\sc\normalsize Métodos Numéricos}\\{\bf Trabajo Práctico Nº1}}
%\author{\begin{tabular}{lcr}Pablo Herrero & LU & \dirmail{pablodherrero@gmail.com}\\Thomas Fischer & 489/08 & \dirmail{tfischer@dc.uba.ar}\\Kevin Allekotte & 490/08 & \dirmail{kevinalle@gmail.com} \end{tabular}}
%\date{\VSP \normalsize{Abril 2010}}
\begin{document}

\materia{Métodos Numéricos}
\titulo{Trabajo Práctico Nº1}
\subtitulo{Errores en serie}
%\grupo{Grupo x}
\integrante{Carla Livorno}{424/08}{carlalivorno@hotmail.com}
\integrante{Mariano De Sousa Bispo}{389/08}{marian\_sabianaa@hotmail.com}

\abstracto{
        El siguiente trabajo se propone analizar teóricamente la propagación de errores de cada una de las series propuestas, 
        en función de la cantidad de dígitos de precisión de la aritmética utilizada para luego contrastar dicho análisis
        con los resultados empíricos.
}

\palabraClave{Error}
\palabraClave{Dígitos}

\begin{titlepage}
\maketitle
\end{titlepage}
\tableofcontents
\newpage

	\begin{section}{Introducción teórica}
		lalala
	\end{section}
	\begin{section}{Desarrollo}
		lalala
	\end{section}
	\begin{section}{Resultados}
		lalala
	\end{section}
	\begin{section}{Discusión}
		lalala
	\end{section}
	\begin{section}{Conclusiones}
		lalala
	\end{section}
	\begin{section}{Apéndices}
		lalala
	\end{section}
	\begin{section}{Referencias}
		lalala
	\end{section}
\end{document}

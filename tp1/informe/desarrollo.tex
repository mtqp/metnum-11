\begin{section}{Desarrollo}
	\begin{subsection}{Explicación}
	\end{subsection}
	\begin{subsection}{Implementación}
		La implementación se realizó en C++. Al necesitar manejar presición arbitraria los tipos de datos nativos del lenguaje no satisfacian nuestras necesidades.
		
		Una solución encontrada a este problema, fue implementar una clase con esta funcionalidad. La clase $Real$ implementa los operadores básicos: suma, resta, multiplicación. división y asignación, los observadores sobre la presición y truncamiento, y funciones para visualización de la información del $Real$ en stdout. Se exporta además dos métodos con funcionalidad importante, uno de ellos para convertir el $Real$ en $double$ (tipo nativo de C++) y otro método que dado un $double$ refresca esta información en la instancia $Real$ correspondiente.
		
		Como precondición de este trabajo práctico la presición máxima de digitos (en base 2) del cálculo de $\pi$ debe ser a lo sumo 51 bits.
		Sabiendo que C++ cumple con el standar IEEE-754, el cual establece 52 bits de mantisa para un tipo de datos $double$, utilizamos a este como simiento de nuestra clase $Real$. Las operaciones aritméticas sobre $Real$ consisten en convertirlos en $double$, realizar la operación correspondiente y transformar el resultado nuevamente en un $Real$, aplicancole previamente el algoritmo de truncamiento o redondeo según corresponda acorde a los parámetros de entrada del programa (cantidad de digitos).
		
		Fuera de la clase implementamos el cálculo de raíz cuadrada, arco tangente y exponenciación de $Real$. Se tomó esta decisión ya que consideramos que estas no son operaciones básicas, por lo que el uso de estas sólo tiene sentido en instancias de $Real$ en situaciones complejas (el cálculo de $\pi$). Tomamos como modelo la implementación de estas operaciones en C++ que requiere la inclusión de una biblioteca (cmath).
		
		//BUCAR EL LUGAR EN EL MUNDO, PARRAFO DESUBICADO!!!
		
		Una instancia de $Real$ se puede generar sólo a partir de un entero de 64 bits con signo, esta decisión se sustenta en la base de que las cuentas en número flotante poseen error. La implementación de todos los algoritmos de cálculo de $\pi$ maximizan el uso de variables de tipo entero. Cuando los calculos dejan de tener sentido en el mundo de los enteros o su presición no es suficiente, se crea una instancia de $Real$ que represente al valor entero y se continúa operando sobre este tipo.
		   
	   A continuación detallamos las optimizaciones de los algoritmos:

		\begin{itemize}
			\item \underline{Gregory:} Refactorizamos nuestra implementación original de la serie que consistía en respetar el orden de los sumandos según el enunciado como las operaciones internas de cada uno de ellos. A continuación se detalla en lenguaje matematico los cambios realizados.
							
				$\sum_{n=0}^{\infty}\frac{(-1)^n}{2n+1} = \sum_{n \in \mathbb{N} / n\;mod(2)=0}^{\infty}\frac{1}{2n+1}-\sum_{n \in \mathbb{N} / n\;mod(2)=1}^{\infty}\frac{1}{2n+1}$
				
				Al hacerlo de esta manera en la ejecución de cada iteración del algoritmo evitamos el calculo de una potencia al costo de almacenar dos números que acumulan las sumas.
				
				Por otro lado, decidimos calcular $2n+1$ en un tipo de dato entero de 64 bits sin signo ya que de esta forma podemos evitar el error en el cálculo (a diferencia de si se realizara con números flotantes) y sería necesario un $n$ muy grande (cantidad de sumandos de la serie) para producir $overflow$.
				
			\item \underline{Machin:} Este algoritmo ejecuta la función $arctan$ la cual realiza la optimización antes mencionada.
			
			\item \underline{Ramanujan:} Este algoritmo fue rediseñado en varias oportunidades al observar diferentes problemas en cada implementación que realizamos. Entre sus operaciones encontramos la función factorial la cual en pocas iteraciones (en enteros sin signo de 64 bits) producia $overflow$. Sabiendo que el rango de valores representable en $double$, y por consiguiente en $Real$ es mayor al de cualquier entero de 64 bits, optamos por realizar la operación factorial en nuestro tipo de datos ($Real$).
			
		No olvidar que la cantidad de valores representable en $double$ y enteros de 64 bits es la misma, ya que las permutaciones binarias posibles de ambos es igual a $2^{64}-1$.\\
		
		Nuestra primer aproximación fue definir la función factorial, la cual dado un $n$ calculaba la $\prod_{k=1}^{n}{k}$. Observamos que la cantidad de operaciones elementales que factorial realiza en cada llamada es lineal en función de $n$. El uso que se le da en el algoritmo de $Ramanujan$ es calcular el factorial de números consecutivos pudiendose asi reutilizar el resultado de la iteración previa.\VSP
		
		A continuación se muestra un breve pseudcódigo del uso de la función factorial en el algoritmo de $Ramanujan$.
		
		$\func{Ramanujan}{n}$\\
		\tab\FOR j=0 \TO n\\
		\tab\tab$factorial(4*j)$\\
		\tab\tab\tab.\\
		\tab\tab\tab.\\
		\tab\tab\tab.\\
		\tab\END
		
		\VSP
		
		Sabiendo que factorial es \Ode{n} y la cantidad de iteraciones del ciclo $for$ es $n$ y suponiendo que el resto de las operaciones del ciclo son constantes el algoritmo tiene complejidad cuadratica en función de $n$.\VSP

		A continuación se muestra un breve pseudocódigo del cálculo del factorial acumulando en una variable el resultado de la iteración anterior.\VSP
		
		$\func{Ramanujan}{n}$\\
		\tab $acum\_fact=1$\\
		\tab\FOR j=1 \TO n\\
		\tab\tab $i = 4*j$\\
		\tab\tab$acum\_fact = acum\_fact*(i-1)*(i-2)*(i-3)*i$\\
		\tab\tab\tab.\\
		\tab\tab\tab.\\
		\tab\tab\tab.\\
		\tab\END
		
		\VSP
		
		De esta forma, se realiza una cantidad constante de multiplicaciones en cada iteración del ciclo $for$ para el calculo del factorial (suponiendo el resto de las operaciones del ciclo de costo constante) el algoritmo tiene complejidad lineal en función de $n$.
		
		La complejidad final del algoritmo de $Ramanujan$ no es esta ya que entre las operaciones del ciclo $for$ se encuentra la operación potencia con costo lineal en función del exponente ($4*j$). La complejidad es de \Ode{n^2}.
		
		Cabe mencionar que el primer término de la serie es un valor constante y entero, por lo que decidimos excluirlo del cálculo de la sumatoria y adicionarlo al final (por ese motivo la implementación final itera de 1 a n).
		\end{itemize}
		
		Para facilitar el desarrollo del programa se generaron varias funciones auxiliares de impresión de información. A la hora de hacer los test no era suficiente la visualización provista por la salida standard de C++. Estas visualizaciones como por ejemplo mostrar el $double$ bit a bit facilitaron la comprensión y analisis de errores programaticos.
		
	\end{subsection}
\end{section}

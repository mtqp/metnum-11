\begin{section}{Apéndices}
	\begin{subsection}{Apéndice A: Enunciado}
		\parindent = 0 pt
\parskip = 11 pt
%\usepackage[width=15.3cm, left=3.5cm, top=2.5cm, height= 24.5cm]{geometry}

\pagestyle{empty}

\newcommand{\real}{\ensuremath{\mathbb{R}}}

%\begin{document}

\begin{centering}
\bf Laboratorio de M\'etodos Num\'ericos - Primer cuatrimestre de 2011 \\
Trabajo Pr\'actico N\'umero 3: CAD - M\'as que splines \\
\end{centering}

\vskip 25pt
\hrule
\vskip 11pt

% {\bf Enunciado}

Los programas de dise\~no asistido por computadora (CAD) son herramientas fundamentales para ingenieros, arquitectos, dise\~nadores, artistas y animadores. Sus interfaces gr\'aficas esconden un sinn\'umero de complicadas operaciones. Un ejemplo b\'asico de tales operaciones es la tarea de seleccionar un punto cualquiera de una curva y moverlo a una nueva posici\'on deformando la curva. El punto seleccionado se ingresa usualmente mediante un dispositivo apuntador (\emph{mouse} o tableta digitalizadora) interactuando con la interfaz, y en general el punto ingresado est\'a meramente \emph{cerca} de la curva.

En este trabajo pr\'actico se deber\'an dise\~nar algoritmos e implementar un programa que, dadas las coordenadas $(\bar{x}_i,\bar{y}_i) \in \real^2$ de una serie de puntos de control ($i = 1..n$), construya una spline natural\footnote{Esto es suficiente para nuestro TP, pero en realidad los sistemas de CAD utilizan m\'as frecuentemente otros mecanismos para obtener, describir y manipular curvas y superficies.} param\'etrica que pase por los puntos en el orden dado. Adem\'as, dadas las coordenadas $(x^*,y^*) \in \real^2$ de un punto cercano a la curva, calcule el punto de la curva m\'as pr\'oximo y construya una nueva spline natural resultante de modificar la spline original de forma que ahora adem\'as pase por la nueva posici\'on $(\bar{x}^*,\bar{y}^*) \in \real^2$ del punto seleccionado. 
El procedimiento descripto se muestra en la figura con la spline original dibujada en l\'inea de trazos y la spline deformada en l\'inea continua.
%\begin{center}
 %\includegraphics[height=0.35\textheight]{./figura2.png}
%\end{center}

El programa deber\'a trabajar las splines como curvas param\'etricas en $\real^2$. Dadas las coordenadas de los puntos de control existen varias estrategias para definir la parametrizaci\'on. Algunas de las parametrizaciones com\'unmente utilizadas son:
\vspace{-15pt}
\begin{description}
  \setlength{\itemsep}{0pt}
  \setlength{\parskip}{0pt}
  \setlength{\parsep}{0pt}
 \item[Uniforme:] la variaci\'on del par\'ametro es igual entre cualquier par de puntos de control consecutivos;
 \item[\emph{Chord-length}:] la variaci\'on del par\'ametro entre dos puntos de control consecutivos es proporcional a la distancia entre los mismos;
 \item[Centr\'ipeta:] la variaci\'on del par\'ametro es proporcional a la ra\'iz cuadrada de la distancia entre los puntos de control\footnote{Este m\'etodo fue propuesto por Eugene Lee en \emph{Choosing nodes in parametric curve interpolation}, Computer-Aided Design 21, 1989.}.
\end{description}
\vspace{-15pt}
En este trabajo pr\'actico deber\'an utilizar alguna de estas parametrizaciones. Opcionalmente podr\'an implementar las restantes y comparar los resultados obtenidos con las tres variantes.

Adem\'as, el programa deber\'a conservar el valor del par\'ametro que le corresponde a cada punto de control y al punto seleccionado, antes y despu\'es de moverlo. 

{\bf Preguntas:}
\vspace{-15pt}
\begin{enumerate}
  \setlength{\itemsep}{0pt}
  \setlength{\parskip}{0pt}
  \setlength{\parsep}{0pt}
 \item >Depende la forma de la curva de la elecci\'on de la parametrizaci\'on? 
 \item >Cambia la forma de la curva si en lugar de deformar la curva conservando la parametrizaci\'on el programa la recalcula al mover el punto?
 \item (Opcional) >C\'omo cambia la forma de la curva seg\'un la condici\'on de borde usada (natural, sujeto, \textit{not-a-knot}, etc.)? 
 \item (Opcional) Si se quiere redibujar cont\'inuamente la curva mientras el usuario mueve el punto seleccionado, >c\'omo se puede calcular esto m\'as eficientemente?
 \item (Opcional) Luego de mover el punto seleccionado, >cambia toda la curva (\emph{control global}) o solamente una parte (\emph{control local})? >Qu\'e consecuencias puede tener esto?
 \item (Opcional) Si el intervalo del par\'ametro se muestrea uniformemente, >los puntos resultantes quedan espaciados uniformemente? >Qu\'e otras alternativas de muestreo ser\'ian apropiadas? 
 \item (Opcional) Si se necesitara que las longitudes de curva entre puntos consecutivos sean todas iguales, >c\'omo deber\'ia muestrearse?
\end{enumerate}

{\bf Archivos de entrada / salida}

La entrada de datos se realizar\'a mediante un archivo de texto con el siguiente formato:
\vspace{-15pt}
\begin{itemize}
  \setlength{\itemsep}{0pt}
  \setlength{\parskip}{0pt}
  \setlength{\parsep}{0pt}
\item En la primera l\'inea figurar\'a el n\'umero $n$ de puntos de control utilizados para definir la spline y, separado por espacio, el n\'umero $m$ de puntos de muestreo de la spline. 
\item En las siguientes $n$ l\'ineas figurar\'an las coordenadas $\bar{x}$ e $\bar{y}$ de cada punto de control separadas por espacio.
\item Una l\'inea en blanco
\item Una l\'inea con las coordenadas $x^*$ e $y^*$ del punto pr\'oximo a la curva, separadas por espacio.
\item Una l\'inea en blanco
\item Una l\'inea con las coordenadas $\bar{x}^*$ e $\bar{y}^*$ de la nueva posici\'on del punto, separadas por espacio.
\end{itemize}

La salida de datos estar\'a dada por un archivo de texto con el siguiente formato:
\vspace{-15pt}
\begin{itemize}
  \setlength{\itemsep}{0pt}
  \setlength{\parskip}{0pt}
  \setlength{\parsep}{0pt}
\item En la primera l\'inea figurar\'a el n\'umero $m$ de puntos muestreados. 
\item En las siguientes $m$ l\'ineas figurar\'an las coordenadas $x$ e $y$ de cada punto muestreado en la spline original, separadas por espacio. Estos puntos corresponder\'an a un muestreo uniforme del rango del par\'ametro e incluir\'an los extremos. De esta forma, probablemente este conjunto de puntos no incluya los puntos de control originales.
\item Una l\'inea en blanco
\item Una l\'inea con las coordenadas del punto en la curva original m\'as pr\'oximo al punto ingresado, separadas por espacio.
\item Una l\'inea en blanco
\item En las siguientes $m$ l\'ineas figurar\'an las coordenadas $x$ e $y$ de cada punto muestreado en la spline deformada, separadas por espacio.
\end{itemize}


\vskip 30pt
\hrule
\vskip 11pt

{\bf Entregas parciales}
\vspace{-15pt}
\begin{description}
  \setlength{\itemsep}{0pt}
  \setlength{\parskip}{0pt}
  \setlength{\parsep}{0pt}
 \item[10 de junio:] Implementaci\'on del c\'alculo de splines y funciones de entrada/salida.
 \item[17 de junio:] Implementaci\'on completa y verificaci\'on.
\end{description}

{\bf Entrega Final}
\vspace{-15pt}
\begin{description}
  \setlength{\itemsep}{0pt}
  \setlength{\parskip}{0pt}
  \setlength{\parsep}{0pt}
 \item[Formato Electr\'onico:] 23 de junio de 2011, hasta las 23:59 hs, a la direcci\'on: 

  {\emph{metnum.lab2011@gmail.com}}
 \item[Formato f\'isico:] 24 de junio de 2011, de 17 a 21 hs.
\end{description}

%\end{document}


	\end{subsection}
	\begin{subsection}{Apéndice B: Código fuente}
		\begin{subsection}{PI: Modo de compilación}
			Para compilar se hace uso de la herramienta Makefile.
			
			Abrir una terminal dentro la carpeta $code$ entregada y escribir el comando "make", pulsar enter.
			
			NOTA:
			
			PI fue programado en plataforma Linux.
		\end{subsection}	
		\begin{subsection}{PI: Modo de uso}
			Una vez compilado escriba en la terminal (posicinado sobre la misma ruta en la que lo compiló) $./pi \; --help$ para recibir ayuda detalla sobre como utilizarlo o $./pi$ para un resumen de la misma.
			
			Recibe al menos dos parámetros para funcionar, el primero el cual selecciona cuál será el modo de uso y el segundo qué algoritmos utilizará. Llamándolo con estos únicos dos parámetros, mostrará los valores por defecto utilizados para ser los cálculos y realizará la ejecución, siendo estos modificables ingresándolos como parámetros de entrada.

			A continuación se escribe el esquema sobre la utilización del programa, recordar que los primeros dos valores encerrados entre corchetes son obligatorios:\\
			'./pi [modoUso] [metodo] [tdigitos] [cantidadIteraciones] [trunca?]'\\	
					
			Por defecto el programa imprime los resultados por la salida estándar, si se desea por ejemplo, guardar los resultados obtenidos en un archivo de texto puede escribir la siguiente sentencia.\\
			
			'./pi [modoUso] [metodo] [tdigitos] [cantidadIteraciones] [trunca?] $>>$ [ruta/nombreArchivo]'\\
			NOTA:
			
			El operador $>>$ utilizado en la consola para redigir la salida estándar a un archivo, es un programa propio de Linux, no siendo este implementado por nosotros.
			
		\end{subsection}	
	\end{subsection}	
\end{section}

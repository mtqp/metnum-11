\begin{section}{Introducción teórica}
	Cuando trabajamos con números reales en una computadora,
	nos encontramos con algunos factores limitantes para nuestros cálculos
	como lo son la aritmética finita y
	también el tipo de representación que se nos provee para trabajar.

	Esto surge como una consecuencia de la necesidad
	de una memoria física finita de la computadora
	en contraste con la precisión infinita
	que requieren la mayoría de los números reales.

	Luego, al intentar representar números en una computadora,
	cuya precisión va mas allá de la otorgada por la misma,
	surgen en la representación del número pequeños errores.
	Tambien se propagan los errores de redondeo y truncamiento debidos al almacenamiento a lo largo 
	de las diferentes operaciones a que sometemos a los números.
	Estos errores, como muestra este trabajo, no son despreciables
	al realizarse ciertos tipos de cálculos.
	
	Por otro lado, el álgebra de precisión finita es diferente al álgebra normal, la propiedad asociativa
	de los números reales no se cumple en general para los datos en punto flotante debido al error de redondeo.
	Si los valores a sumar son de muy diferente orden de magnitud se producirá un resultado más aproximado al
	correcto si se suman ordenadamente de menor a mayor. Esto nos demuestra que el orden en que se ejecutan las
	operaciones es importante para un computador.
\end{section}
